\documentclass[10pt,a4paper]{beamer} %ppt模式

\usepackage{ctex}

\usepackage{beamerthemesplit} % 加载主题宏包
\usetheme{Warsaw} % 选用该主题

\usepackage[T1]{fontenc}

%插入图片, 写公式, 画表格等
\usepackage{subfig}
\usepackage{amssymb,amsmath,mathtools}
\usepackage{amsfonts,booktabs}
\usepackage{lmodern,textcomp}
\usepackage{color}
\usepackage{tikz}
\usepackage[utf8]{inputenc}
\usepackage{natbib}
\usepackage{multicol}
\usepackage{graphicx}
\setbeamertemplate{footline}{}%清除底部信息,更美观

\begin{document}
	
	%基本信息
	\title{袁世凯成功窃取革命果实的原因}
	\subtitle{近纲课}
	\author{林海轩}
	\institute{复旦大学物理学系}
	\date{}
	
	%生成标题页
	\begin{frame}
		\titlepage
	\end{frame}
	
	%目录页,代码无需改动
	\begin{frame}
		\tableofcontents
	\end{frame}
	
	
	%%%%%%%%%%%%%%%%%%%%%%%%%%%%%%%%%%%%%%%%%%%%%%%%%%%%%%%%%%%%%%%%%%%%%%%%%%%%%%%%%%%%%%%%%%%%%%%
	
	\begin{frame}
		1911年的武昌起义如星星之火,把整片中华大地燃烧起来了,各省纷纷脱离清政府统治,革命形势一片巨好,可没过多久,中华民国总统孙中山竟然让位给军阀袁世凯,这中间发生了什么,袁世凯又是如何窃取革命果实的呢?
	\end{frame}
	
	\section{袁世凯军事实力强大}
	\begin{frame}
		武昌起义爆发之后,清政府如同热锅上的蚂蚁,当时清政府官员中能镇压革命军的只有袁世凯一人,因为当时新军基本上在袁世凯控制中,武昌起义给了袁世凯机会,他明白这是掌握军政大权的时机,于是他狮子大开口向清皇朝提出六个条件分别是:
		
		1:明年召开国会。
		
		2:组织责任内阁。
		
		3:开放党禁。
		
		4:宽容武昌起义的人员。
		
		5:袁世凯要前线一切指挥权。
		
		6:粮响要充足。
		
		这六个条件的基本内容就是要清政府交给他军事和政治的全部权力。清政府迫于形势严峻,不得不答应袁世凯的条件,就这样袁世凯掌握了清政府的军政大权,所以如果袁世凯不承认中华民国,那么以中华民国的实力是不可能统一整个国家,而且还可能被清军剿灭,所以孙中山不得不考虑交出权力,保证共和。
		
		
	\end{frame}
	
	\section{帝国主义的支持}
	\begin{frame}
		武昌起义胜利,中华民国军政府成立以后,军政府向西方帝国主义发布了一个文件,大概内容就是各国支援清政府的物质一概没收,还有清政府与帝国主义签订的条约,中华民国全部不承认。这个文件被各国知道以后,一下子就愤怒了,以至于对这个新生的政权没好感,只能去支持清政府了,而英国驻中国公使又是袁世凯的好朋友,袁世凯又讨好列强,所以他们就非常支持袁世凯,有了西方列强的支持,袁世凯更加有持无恐。军政府下发的这个文件,合情合理,只是太操之过急,革命局势还不稳固,这个时候去冒犯列强,得不偿失。
	\end{frame}
	
	\section{外强中干的中华民国}
	\begin{frame}
		武昌起义之后,十多个省宣布独立,可这十多个省里面听孙中山的没几个,因为大部分的省都被军阀和一些大地主控制着,还有一些直接就被清政府官员控制,比如江西省当时实际控制人是清朝巡抚冯汝骙,他们压根就不懂什么事民主,只知道争权夺利,所以中华民国就是外强中干的,而且武昌起义的指挥官黎元洪一直和袁世凯眉来眼去的,所以中华民国根本没有能力抵抗袁世凯的清军。
	\end{frame}
	
	\section{孙中山思想的单纯}
	\begin{frame}
		南京临时政府成立以后,面临的第一个问题就是财政问题,因为没钱,军队的伙食已经从干饭变成稀饭,而且有些时候连喝粥都不能保障,所以只能遣返军队,孙中山这时已经没有亲兵了如何和袁世凯争权呢?而且再加上袁世凯答应了条件,逼皇帝退位,这时孙中山只能幻想袁世凯能带领中华民国走向富强。
		
		
	\end{frame}
	%%%%%%%%%%%%%%%%%%%%%%%%%%%%%%%%%%%%%%%%%%%%%%%%%%%%%%%%%%%%%%%%%%%%%%%%%%%%%%%%%%%%%%%%%%%%%%%	
	
	%参考文献(非调用.bib文件,而是手动输入)
	\appendix
	\begin{frame}{参考文献}
		\begin{thebibliography}{99} % 最大可能的参考文献数目,可以根据实际情况调整
			\bibitem[Author et al., 2021]{ref1}
			Dewey. Art as Experience[J]. 高等教育出版社, 11.4(2022):1-4.
			\newblock Title of the first reference.
			\newblock \emph{Journal Name}, \emph{Volume}(Issue), PageRange.
			
			\bibitem[Another Author, 2022]{ref2}
			Plato. Utopia[M]. 高等教育出版社, 01(2333):-2-4.
			\newblock Title of the second reference.
			\newblock \emph{Another Journal}, \emph{Volume}(Issue), PageRange.
		\end{thebibliography}
		\bibliographystyle{plain}
		\bibliography{ypzz}
	\end{frame}
	
	\begin{frame}[plain,c]
		\begin{center}
			\Huge 感谢聆听 !
		\end{center}
	\end{frame}
	
\end{document}