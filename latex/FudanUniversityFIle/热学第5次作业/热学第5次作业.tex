\documentclass[a4paper,11pt]{amsart}

\usepackage{gensymb}

\usepackage{tikz}
\usetikzlibrary{arrows.meta}
\usepackage{caption}

\usepackage{times}
\usepackage[top=27mm, left=23mm, bottom=23mm, right=23mm]{geometry}
\usepackage{amsfonts, amssymb, amsgen, amsthm, amscd, amsmath}
\usepackage{mathtools}
\mathtoolsset{showonlyrefs=true}

\usepackage{bm}
%\usepackage{mathpazo}
\usepackage{domitian}
\usepackage[T1]{fontenc}
\let\oldstylenums\oldstyle

\usepackage{enumerate}
\usepackage{color}
\usepackage[all]{xy}

\newtheorem{theorem}{Theorem}[section]
\newtheorem{proposition}[theorem]{Proposition}
\newtheorem{lemma}[theorem]{Lemma}
\newtheorem{corollary}[theorem]{Corollary}
\newtheorem{claim}[theorem]{Claim}
\theoremstyle{definition}
\newtheorem{remark}[theorem]{Remark}
\newtheorem{example}[theorem]{Example}
\newtheorem{definition}[theorem]{Definition}

\newcommand{\outimes}[2]{\overset{#1}{\underset{#2}{\otimes}}}
\newcommand{\C}[1]{\mathcal{#1}}
\newcommand{\B}[1]{\mathbb{#1}}
\newcommand{\G}[1]{\mathfrak{#1}}
\newcommand{\rmod}[1]{\text{{\bf Mod}-}{#1}}

\newcommand{\Span}{\text{\rm Span}}
\newcommand{\Tor}{\text{\rm Tor}}
\newcommand{\Ind}{\text{\rm Ind}}
\newcommand{\Res}{\text{\rm Res}}
\newcommand{\Ext}{\text{\rm Ext}}
\newcommand{\Hom}{\text{\rm Hom}}
\newcommand{\CoInd}{\text{\rm CoInd}}
\newcommand{\Simp}{{\Delta}}
\newcommand{\Diff}{{\Omega}}
\newcommand{\xla}[1]{\xleftarrow{#1}}
\newcommand{\colim}{\text{colim}}
\renewcommand{\baselinestretch}{1.15}

\setlength{\parskip}{1.2mm}
\setlength{\parindent}{0mm}

\title{Thermodynamics Assignment For The Fifth Time}

\author{Haixuan Lin - 23307110267}
\email{23307110267@m.fudan.edu.cn}


\address{Fudan University, Physics Department, China}

\begin{document}
	
	\begin{abstract}
		Here is the thermodynamics assignment for the fifth time which is for the course given by professor Yuanbo Zhang. In order to practise the expertise in scientific film of physics, students need to practise using \LaTeX to composing their own work, even if this is only a ordinary homework.
	\end{abstract}
	
	\maketitle
	\section*{Main Text}
	
	\subsection*{4.4.2}
	
	\subsubsection*{(1)}
	
	This is a quasi-static process, using the van der Waals equation to approximate the relationship between volume and pressure
	
	$$
	A=\int_{V_1}^{V_2}{\left( \frac{RT}{V_m-b}-\frac{a}{V_{m}^{2}} \right) \mathrm{d}V_m=RT\ln \frac{V_{2,m}-b}{V_{1,m}-b}+\frac{a}{V_2,m}}-\frac{a}{V_{1,m}}
	$$
	
	\subsubsection*{(2)}
	
	Since it is an equal volume process, there is no work, so the heat absorption is equal to the change in internal energy.
	
	$$
	Q=\Delta U=c\Delta T
	$$
	
	\subsection*{4.4.6}
	
	\subsubsection*{(1)}
	
	Apply the definition of enthalpy directly to the formula given in the stem of the problem.
	
	$$
	H_m=U_m+pV_m=cT-apT+pV_{0,m}+apT+bT^2=cT+pV_{0,m}+bT^2
	$$
	
	\subsubsection*{(2)}
	
	According to various definitions of heat capacity.
	
	$$
	C_{V,m}=\frac{\mathrm{d}U_m}{\mathrm{d}T}=c-ap-aT\left( \frac{\partial p}{\partial T} \right) _{V_m}=c-a\frac{V_m-V_{0,m}-aT}{b}+\frac{aT^2}{b}=c-\frac{aV}{b}+\frac{aV_{0,m}}{b}+\frac{2a^2T}{b}
	$$
	
	\subsection*{4.4.8}
	
	The energy contribution per mole of the reaction is
	
	$$
	W_e=4eN_{\mathrm{A}}U
	$$
	
	So
	
	$$
	\eta =\left| \frac{W_e}{\Delta H} \right|=82.8\%
	$$
	
	
	\subsection*{4.5.2}
	
	\subsubsection*{(1)}
	
	The internal energy of an ideal gas depends only on temperature, so the change in internal energy is zero.
	
	$$
	\Delta U=0
	$$
	
	According to the first law of thermodynamics, we have
	
	$$
	Q=A=-\int_{V_1}^{V_2}{p\mathrm{d}V=-\frac{m}{M^{\mathrm{mol}}}RT\ln 2}=-7862\,\mathrm{J}
	$$
	
	This means that the outside world is doing 7862 joules of work on the gas, and the gas is giving off 7862 joules of heat to the environment in order to maintain its internal energy.
	
	\subsubsection*{(2)}
	
	Adiabatic equation.
	
	$$
	pV^{\gamma}=C
	$$
	
	First law of thermodynamics with $Q=0$.
	
	$$
	\Delta U=W=-\int_{V_1}^{V_2}{p\mathrm{d}V}=-\int_{V_1}^{V_2}{\frac{C}{V^{\gamma}}\mathrm{d}V=\frac{1}{1-\gamma}C}\left. V^{1-\gamma} \right|_{V_1}^{V_2}=\frac{1}{1-\gamma}C\left( V_{2}^{1-\gamma}-V_{1}^{1-\gamma} \right) 
	$$
	
	Some algebraic deformation.
	
	$$
	pV\cdot V^{\gamma -1}=\frac{m}{M^{\mathrm{mol}}}RT\cdot V^{\gamma -1}=C
	$$
	
	$$
	\frac{m}{M^{\mathrm{mol}}}RT=CV^{1-\gamma}
	$$
	
	What's more:
	
	$$
	\gamma =\frac{C_{V,m}+R}{C_{V,m}}=\frac{7}{5}
	$$
	
	As a result:
	
	$$
	\Delta U=W=9061\,\mathrm{J}
	$$
	
	This means that the heat absorbed by the gas from the outside is 0, but because the outside world does 9061 joules on the gas, the internal energy of the gas increases by 9060 joules. 
	
	\subsubsection*{(3)}
	
	Pressure is a constant.
	
	$$
	W=\frac{1}{2}pV=\dfrac{1}{2}\dfrac{m}{M^{\mathrm{mol}}}RT=5.6\times 10^3\,\,\mathrm{J}
	$$
	
	Utilizing the isobaric heat capacity
	
	$$
	Q=C_{p,m}\frac{m}{M^{\mathrm{mol}}}\Delta T=C_{p,m}\frac{m}{M^{\mathrm{mol}}}\frac{-pV}{2m/M^{\mathrm{mol}}R}=-\frac{C_{p,m}}{R}W=-1.97\times 10^4\,\mathrm{J}
	$$
	
	By the first law of thermodynamics
	
	$$
	\Delta U=W+Q=-1.41\times 10^4\,\mathrm{J}
	$$
	
	\subsection*{4.5.8}
	
	According to the adiabatic equation and the ideal gas equation of state.
	
	$$
	pV^{\gamma}=C
	$$
	
	So
	
	$$
	\frac{\mathrm{d}p}{p}+\gamma \frac{\mathrm{d}V}{V}=0
	$$
	
	And
	
	$$
	pV=nRT
	$$
	
	So
	
	$$
	\frac{\mathrm{d}p}{p}+\frac{\mathrm{d}V}{V}=\dfrac{\mathrm{d}T}{T}
	$$
	
	As a result
	
	$$
	\frac{\mathrm{d}T}{T}=\frac{\gamma -1}{\gamma}\frac{\mathrm{d}p}{p}
	$$
	
	Then
	
	$$
	T=T_0\left( \frac{p}{p_0} \right) ^{\frac{\gamma -1}{\gamma}}
	$$
	
	Take it into original equation
	
	$$
	p^{-\frac{1}{\gamma}}\mathrm{d}p=-\frac{Mg}{RT_0}p_{0}^{\frac{\gamma -1}{\gamma}}\mathrm{d}z
	$$
	
	$$
	\frac{\gamma}{\gamma -1}p^{\frac{\gamma -1}{\gamma}}-\frac{\gamma}{\gamma -1}{p_0}^{\frac{\gamma -1}{\gamma}}=-\frac{Mgh}{RT_0}p_{0}^{\frac{\gamma -1}{\gamma}}
	$$
	
	$$
	p^{\frac{\gamma -1}{\gamma}}=p_{0}^{\frac{\gamma -1}{\gamma}}\left( 1-\frac{\gamma -1}{\gamma}\frac{Mgh}{RT_0} \right) 
	$$
	
	$$
	p=p_0\left( 1-\frac{Mgh}{C_{p,m}T_0} \right) ^{\frac{\gamma}{\gamma -1}}
	$$
	
	\subsection*{4.5.9}
	
	Take the relationship into the ideal gas equation.
	
	$$
	\frac{a_{0}^{2}}{V}=nRT
	$$
	
	Thus
	
	$$
	C=\frac{\mathrm{d}Q}{\mathrm{d}T}=\frac{\mathrm{d}U-\mathrm{d}W}{\mathrm{d}T}=C-\frac{\mathrm{d}W}{\mathrm{d}T}=C+\frac{p\mathrm{d}V}{\mathrm{d}T}=C-\frac{pa_{0}^{2}}{nRT^2}=C-\frac{a_{0}^{2}}{TV}
	$$
	
	\subsection*{4.5.13}
	
	According to the adiabatic equation:
	
	$$
	\frac{\mathrm{d}p}{p}+\gamma \frac{\mathrm{d}V}{V}=0
	$$
	
	$$
	\mathrm{d}F=A\mathrm{d}p=-\gamma \frac{pA^2}{V_0}\mathrm{d}x=-\gamma \frac{\left( mg/A+p_0 \right) A^2}{V_0}\mathrm{d}x
	$$
	
	And we have
	
	$$
	\nu =\frac{1}{2\pi}\sqrt{\frac{\mathcal{K}}{m}}=\frac{1}{2\pi}\sqrt{\frac{1}{m}\gamma \frac{\left( mg/A+p_0 \right) A^2}{V_0}}
	$$
	
	$$
	\gamma =\frac{4\pi ^2\nu ^2mV_0}{mgA+p_0A^2}
	$$
	
	\subsection*{4.6.3}
	
	\subsubsection*{(1)}
	
	The ideal gas has some properties like:
	
	$$
	U=\frac{i}{2}kTN=\frac{3}{2}\frac{R}{N_{\mathrm{A}}}TnN_{\mathrm{A}}=\frac{3}{2}nRT=\frac{3}{2}pV
	$$
	
	Base on that
	
	$$
	Q_{a\rightarrow b}=\Delta U_{a\rightarrow b}-W_{a\rightarrow b}=\frac{3}{2}\left( 9-1 \right) p_0V_0-0=12p_0V_0
	$$
	
	$$
	Q_{b\rightarrow c}=\Delta U_{b\rightarrow c}-W_{b\rightarrow c}=\frac{3}{2}\left( 27-9 \right) p_0V_0+18p_0V_0=45p_0V_0
	$$
	
	$$
	Q_{c\rightarrow a}=\Delta U_{c\rightarrow a}-W_{c\rightarrow a}=\frac{3}{2}\left( 1-27 \right) p_0V_0+\int_{3V_0}^{V_0}{\frac{p_0}{V_{0}^{2}}V^2\mathrm{d}V}=-\frac{143}{3}p_0V_0
	$$
	
	\subsubsection*{(2)}
	
	Base on (1)
	
	$$
	W^*=W_{b\rightarrow c}+W_{c\rightarrow a}=\dfrac{28}{3}p_0V_0
	$$
	
	$$
	Q^*=Q_{a\rightarrow b}+Q_{b\rightarrow c}=57p_0V_0
	$$
	
	So
	
	$$
	\eta=\dfrac{W^*}{Q^*}=16.4%
	$$
	
	\subsection*{4.7.3}
	
	\subsubsection*{(1)}
	
	Heat is extracted from the low temperature heat source, and the heat generated by itself is transferred to the high temperature heat source. By inference of the Carnot chiller:
	
	$$
	W+\dfrac{T_0}{T_1-T_0}W=a(T_1-T_0)
	$$
	
	Give up the smaller solution we have
	
	$$
	T_1=T_0+\frac{W}{2a}\left( 1+\sqrt{1+\frac{4aT}{W}} \right) 
	$$
	
	\subsubsection*{(2)}
	
	Directly:
	
	$$
	W=a(T_2-T_0)
	$$
	
	So 
	
	$$
	T_2=T_0+\dfrac{W}{a}
	$$
	
	Obiously $T_2<T_1$, thus (1) is better.
	
\end{document}

