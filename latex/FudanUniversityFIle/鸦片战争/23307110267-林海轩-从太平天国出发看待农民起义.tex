\documentclass[10pt,a4paper]{beamer} %ppt模式

\usepackage{ctex}

\usepackage{beamerthemesplit} % 加载主题宏包
\usetheme{Warsaw} % 选用该主题

\usepackage[T1]{fontenc}

%插入图片, 写公式, 画表格等
\usepackage{subfig}
\usepackage{amssymb,amsmath,mathtools}
\usepackage{amsfonts,booktabs}
\usepackage{lmodern,textcomp}
\usepackage{color}
\usepackage{tikz}
\usepackage[utf8]{inputenc}
\usepackage{natbib}
\usepackage{multicol}
\usepackage{graphicx}
\setbeamertemplate{footline}{}%清除底部信息,更美观

\begin{document}
	
	%基本信息
	\title{从太平天国出发看待农民起义}
	\subtitle{近纲第二次课}
	\author{林海轩}
	\institute{复旦大学物理学系}
	\date{}
	
	%生成标题页
	\begin{frame}
		\titlepage
	\end{frame}
	
	%目录页,代码无需改动
	\begin{frame}
		\tableofcontents
	\end{frame}
	
	
	%%%%%%%%%%%%%%%%%%%%%%%%%%%%%%%%%%%%%%%%%%%%%%%%%%%%%%%%%%%%%%%%%%%%%%%%%%%%%%%%%%%%%%%%%%%%%%%
	\section{回顾历史}
	\begin{frame}
		太平天国运动是中国历史上规模巨大、持续时间长久的一次农民起义,对于理解中国乃至世界历史中农民起义的意义和影响提供了丰富的视角和深刻的启示。从1840年至1864年,这场运动不仅挑战了清朝的统治,也在国际舞台上展现了其深远的影响。分析太平天国,可以帮助我们深入理解农民起义的根源、发展过程及其对社会变革的贡献。
	\end{frame}
	
	\section{分析}
	\subsection{国内视角}
	\begin{frame}
		首先,太平天国反映了农民起义根源于社会不公和经济困境。19世纪中叶,清朝政府腐败无能,加之自然灾害频发,农民负担沉重,生活困苦。太平天国运动的领导者洪秀全提出的“天朝田亩制”等改革思想,呼应了农民对土地和财富分配更公平的渴望,这正是农民起义能够迅速获得广泛支持的关键。
		太平天国的财政管理是其试图建立有效治理结构的关键组成部分。太平天国推行的“天朝田亩制”计划,旨在重新分配土地,消除贫富差距。这一政策的实施,理论上能够提高农民的生活水平,通过增加政府的土地收入来加强其财政基础。然而,由于持续的战争和管理上的混乱,加之清政府和西方势力的经济封锁,太平天国的经济和财政资源遭受重创,财政状况日益恶化。
	\end{frame}
	\begin{frame}
		其次,太平天国的发展和失败揭示了农民起义面临的内外挑战。在内部,领导层的分裂、管理无序以及军事战略上的失误导致了最终的失败。在外部,清政府联合西方列强的军事干预对太平天国形成了巨大压力。这说明农民起义要想成功,不仅需要强有力的内部组织和领导,也需要有效应对外部势力的干扰和压制。
		人口管理对于太平天国来说同样是一项巨大的挑战。太平天国在控制区域内实施了一系列社会改革措施,包括推广教育、提倡男女平等等,旨在提升人民的生活质量和社会地位。然而,长期的战争造成了巨大的人口损失和社会动荡,不仅削弱了太平天国的人力资源基础,也影响了其长期的治理能力和社会稳定。
	\end{frame}
	
	\begin{frame}
		再者,太平天国对于社会变革的贡献不容忽视。虽然太平天国未能彻底推翻清政府,但其提出的一系列改革措施,如废除科举、推行土地改革、提倡男女平等等,对中国社会的现代化进程产生了深远影响。这些改革思想预示了近代中国社会变革的方向,激发了后来的改革运动。
	\end{frame}
	\subsection{国际视角}
	\begin{frame}
		从国际视角看,太平天国运动也是全球化背景下农民起义的一个例证。通过与同时期的美国南北战争的联系,可以看出农民起义如何与全球贸易网络相互作用,进而影响国际政治经济格局。太平天国期间,中国与西方国家的关系发生了重大变化,这不仅影响了中国的近代化进程,也改变了世界历史的发展轨迹。
	\end{frame}
	
	%%%%%%%%%%%%%%%%%%%%%%%%%%%%%%%%%%%%%%%%%%%%%%%%%%%%%%%%%%%%%%%%%%%%%%%%%%%%%%%%%%%%%%%%%%%%%%%	
	
	%参考文献(非调用.bib文件,而是手动输入)
	\appendix
	\begin{frame}{参考文献}
		\begin{thebibliography}{99} % 最大可能的参考文献数目,可以根据实际情况调整
			\bibitem[Author et al., 2021]{ref1}
			Dewey. Art as Experience[J]. 高等教育出版社, 11.4(2022):1-4.
			\newblock Title of the first reference.
			\newblock \emph{Journal Name}, \emph{Volume}(Issue), PageRange.
			
			\bibitem[Another Author, 2022]{ref2}
			Plato. Utopia[M]. 高等教育出版社, 01(2333):-2-4.
			\newblock Title of the second reference.
			\newblock \emph{Another Journal}, \emph{Volume}(Issue), PageRange.
		\end{thebibliography}
		\bibliographystyle{plain}
		\bibliography{ypzz}
	\end{frame}
	
	\begin{frame}[plain,c]
		\begin{center}
			\Huge 感谢聆听 !
		\end{center}
	\end{frame}
	
\end{document}