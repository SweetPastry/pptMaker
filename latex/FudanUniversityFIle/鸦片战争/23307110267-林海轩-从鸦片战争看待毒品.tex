\documentclass[10pt,a4paper]{beamer} %ppt模式

\usepackage{ctex}

\usepackage{beamerthemesplit} % 加载主题宏包
\usetheme{Warsaw} % 选用该主题

\usepackage[T1]{fontenc}

%插入图片, 写公式, 画表格等
\usepackage{subfig}
\usepackage{amssymb,amsmath,mathtools}
\usepackage{amsfonts,booktabs}
\usepackage{lmodern,textcomp}
\usepackage{color}
\usepackage{tikz}
\usepackage[utf8]{inputenc}
\usepackage{natbib}
\usepackage{multicol}
\usepackage{graphicx}
\setbeamertemplate{footline}{}%清除底部信息,更美观

\begin{document}
	
	%基本信息
	\title{从鸦片战争的出发看待毒品}
	\subtitle{近纲第一次课}
	\author{林海轩}
	\institute{复旦大学物理学系}
	\date{}
	
	%生成标题页
	\begin{frame}
		\titlepage
	\end{frame}
	
	%目录页,代码无需改动
	\begin{frame}
		\tableofcontents
	\end{frame}
	
	
	%%%%%%%%%%%%%%%%%%%%%%%%%%%%%%%%%%%%%%%%%%%%%%%%%%%%%%%%%%%%%%%%%%%%%%%%%%%%%%%%%%%%%%%%%%%%%%%
	\section{回顾历史}
	\begin{frame}
		\subsection{简述}
		鸦片战争是19世纪中叶发生在中国和英国之间的一系列冲突,主要因为鸦片贸易问题和中国的闭关政策。鸦片战争通常分为两次:第一次鸦片战争(1839-1842年)和第二次鸦片战争(1856-1860年),也称为“箭战”。
	\end{frame}
		
	\begin{frame}
		\subsection{第一次鸦片战争}
		第一次鸦片战争(1839-1842年)
		起因:英国通过非法贸易向中国大量输入鸦片,导致大量白银流出和社会问题加剧。中国清政府为了禁止鸦片贸易,派遣林则徐到广东执行禁烟政策,最著名的行动是虎门销烟,即公开销毁了大量鸦片。
		经过:英国为了保护其贸易利益,发动了对中国的军事攻击。清朝军队落后,无法抵挡英军的进攻。
		结果:双方最终签订了《南京条约》,这是中国历史上第一个不平等条约。条约规定中国赔偿英国巨额战争赔款,开放五个通商口岸与英国进行贸易,并割让香港岛给英国。
	\end{frame}

	\begin{frame}
		\subsection{第二次鸦片战争}
			第二次鸦片战争(1856-1860年)
		起因:第二次鸦片战争又称“箭战”,起因是广州一艘名为“箭号”的中国船只被英国军舰扣押,引发了冲突。
		经过:英法联军参战,对中国进行了更加深入的军事侵略。
		结果:战争以《天津条约》和《北京条约》的签订结束,这些条约进一步加重了中国的不平等地位,开放了更多的城市进行贸易,允许外国传教士在内地传教,赔偿巨额战争赔款等。
		鸦片战争对中国历史产生了深远影响,不仅导致了中国主权和领土的丧失,而且揭示了清政府的腐败和落后,促使中国开始了一系列的自我反省和近代化改革尝试,这些努力最终导致了19世纪末20世纪初的辛亥革命和中国近代化进程的开启。\cite{111}\cite{222}
	\end{frame}
	
	\section{如何做}
		\subsection{分清毒品}
		\begin{frame}
		毒品通常根据它们的化学结构、效果、以及它们对人体的影响来进行区分。以下是一些常见的毒品分类及其特点:
		
		中枢神经系统(CNS)刺激剂:如可卡因、安非他命(包括冰毒)、咖啡因。这些物质可以提高神经系统的活动,导致精神亢奋、能量增加、警觉性提高、食欲减少等。
		
		中枢神经系统抑制剂:如酒精、苯二氮卓类(安眠药、镇静剂)、鸦片类(如吗啡、海洛因)。这类物质通常能减缓大脑活动,导致放松感、减少焦虑、嗜睡、反应时间延长等。
		
		幻觉剂:如LSD、迷幻蘑菇(含有裸盖菇碱)、可卡因的一种形式—裂解可卡因。幻觉剂能改变人的感知、思维、情感状态,产生视觉或听觉幻觉。
		
		大麻:包括大麻花、大麻树脂和大麻油。大麻可以产生放松感、幸福感、时间感和感官知觉的扭曲,但也可能导致焦虑和偏执。
		\end{frame}
		\begin{frame}
			毒品成瘾的机制复杂,涉及大脑多个神经递质系统。以下是一些关键点:
			
			多巴胺系统:多巴胺是与奖励和快感感觉相关的关键神经递质。大多数毒品通过增加大脑奖励系统中的多巴胺水平,增强快感和奖励感,从而促使个体重复使用毒品。
			
			神经适应:随着毒品的持续使用,大脑会对其存在进行适应,导致原有的正常快乐源无法再产生同样的满足感,使得个体为了获得相同的快感或避免不适感而需要更多的毒品。
			
			学习和记忆:毒品使用与特定环境或情绪状态相联系时,这些环境和状态就可能触发强烈的毒品渴求,这是通过大脑的学习和记忆路径,特别是海马体和杏仁核等区域的变化实现的。
			
			刺激-反应习惯:重复毒品使用会在大脑形成强烈的刺激-反应模式,即某些刺激(如见到与毒品使用相关的人、地点、事物)会自动触发使用毒品的冲动,即使是在没有意识到的情况下。
			
			成瘾是一个逐步发展的过程,涉及到遗传、环境、心理社会因素的相互作用。理解这些机制有助于开发更有效的治疗和预防策略。

		\end{frame}
		
		\subsection{必须要严格禁毒}
		\begin{frame}
			历史原因:中国的禁毒政策深受历史的影响,尤其是鸦片战争给中国带来的深重灾难。19世纪中叶,大量鸦片流入中国,导致社会经济动荡,数百万人沉迷于鸦片,严重影响了国家的发展和人民的健康。这段历史给中国社会留下了深刻的印象,使得中国政府在毒品问题上持有非常严格的态度。
			
			社会稳定:毒品问题不仅关系到个体健康,还与社会稳定和安全密切相关。毒品滥用和贩卖可导致犯罪率上升、家庭破碎、青少年迷失方向等社会问题,严重威胁社会秩序和公共安全。中国政府禁毒,旨在保护人民群众的生命安全和身体健康,维护社会稳定和和谐。
			
			经济发展:毒品问题对经济发展有着直接和间接的负面影响。毒品滥用导致劳动力资源的浪费,增加了医疗、法律和社会福利成本,影响经济效率和增长。通过严格禁毒,中国旨在营造良好的经济环境,促进持续健康的经济发展。
			
			公共卫生:毒品滥用是重大的公共卫生问题,与HIV/AIDS、肝炎等传染病传播密切相关。毒品使用者通过共用注射器等方式极大地增加了这些疾病的传播风险。中国严格禁毒,部分原因是为了控制这些疾病的传播,保护公共卫生。
			
			国际责任:随着全球化进程的加深,毒品问题已成为全球性挑战,各国在禁毒工作中需要承担共同责任。中国作为国际社会的重要成员,通过严格禁毒,积极履行国际义务,与国际社会共同努力打击毒品问题。
		\end{frame}

	%%%%%%%%%%%%%%%%%%%%%%%%%%%%%%%%%%%%%%%%%%%%%%%%%%%%%%%%%%%%%%%%%%%%%%%%%%%%%%%%%%%%%%%%%%%%%%%	
	
	%参考文献(非调用.bib文件,而是手动输入)
	\appendix
	\begin{frame}{参考文献}
		\begin{thebibliography}{99} % 最大可能的参考文献数目,可以根据实际情况调整
			\bibitem[Author et al., 2021]{ref1}
			Dewey. Art as Experience[J]. 高等教育出版社, 11.4(2022):1-4.
			\newblock Title of the first reference.
			\newblock \emph{Journal Name}, \emph{Volume}(Issue), PageRange.
			
			\bibitem[Another Author, 2022]{ref2}
			Plato. Utopia[M]. 高等教育出版社, 01(2333):-2-4.
			\newblock Title of the second reference.
			\newblock \emph{Another Journal}, \emph{Volume}(Issue), PageRange.
		\end{thebibliography}
		\bibliographystyle{plain}
		\bibliography{ypzz}
	\end{frame}
	
	\begin{frame}[plain,c]
		\begin{center}
			\Huge 感谢聆听 !
		\end{center}
	\end{frame}
	
\end{document}