
%%%%%%%%%%%%%%%%%%%%%%%%%%%%%%%%%%%%%%%%%%%%%%%%%%%%%%%%
\documentclass[12pt,a4paper]{article}% 文档格式
\usepackage{ctex,hyperref}% 输出汉字
\usepackage{times}% 英文使用Times New Roman
%%%%%%%%%%%%%%%%%%%%%%%%%%%%%%%%%%%%%%%%%%%%%%%%%%%%%%%%
\title{\fontsize{18pt}{27pt}\selectfont% 小四字号,1.5倍行距
	{\heiti% 黑体 
		A Difficult Question From My Friend}}% 题目
%%%%%%%%%%%%%%%%%%%%%%%%%%%%%%%%%%%%%%%%%%%%%%%%%%%%%%%%
\author{\fontsize{12pt}{18pt}\selectfont% 小四字号,1.5倍行距
	{\fangsong% 仿宋
		Haixuan Lin}\\% 标题栏脚注
	\fontsize{10.5pt}{15.75pt}\selectfont% 五号字号,1.5倍行距
	{\fangsong% 仿宋
		(Fudan University department of physics
		)}}% 作者单位,“~”表示空格
%%%%%%%%%%%%%%%%%%%%%%%%%%%%%%%%%%%%%%%%%%%%%%%%%%%%%%%%
\date{}% 日期(这里避免生成日期)
%%%%%%%%%%%%%%%%%%%%%%%%%%%%%%%%%%%%%%%%%%%%%%%%%%%%%%%%
\usepackage{amsmath,amsfonts,amssymb}% 为公式输入创造条件的宏包
\allowdisplaybreaks
%%%%%%%%%%%%%%%%%%%%%%%%%%%%%%%%%%%%%%%%%%%%%%%%%%%%%%%%
\usepackage{graphicx}% 图片插入宏包
\usepackage{subfigure}% 并排子图
\usepackage{float}% 浮动环境,用于调整图片位置
\usepackage[export]{adjustbox}% 防止过宽的图片
%%%%%%%%%%%%%%%%%%%%%%%%%%%%%%%%%%%%%%%%%%%%%%%%%%%%%%%%
\usepackage{bibentry}
\usepackage{natbib}% 以上2个为参考文献宏包
%%%%%%%%%%%%%%%%%%%%%%%%%%%%%%%%%%%%%%%%%%%%%%%%%%%%%%%%
\usepackage{abstract}% 两栏文档,一栏摘要及关键字宏包
\renewcommand{\abstracttextfont}{\fangsong}% 摘要内容字体为仿宋
\renewcommand{\abstractname}{\textbf{摘\quad 要}}% 更改摘要二字的样式
%%%%%%%%%%%%%%%%%%%%%%%%%%%%%%%%%%%%%%%%%%%%%%%%%%%%%%%%
\usepackage{xcolor}% 字体颜色宏包
\newcommand{\red}[1]{\textcolor[rgb]{1.00,0.00,0.00}{#1}}
\newcommand{\blue}[1]{\textcolor[rgb]{0.00,0.00,1.00}{#1}}
\newcommand{\green}[1]{\textcolor[rgb]{0.00,1.00,0.00}{#1}}
\newcommand{\darkblue}[1]
{\textcolor[rgb]{0.00,0.00,0.50}{#1}}
\newcommand{\darkgreen}[1]
{\textcolor[rgb]{0.00,0.37,0.00}{#1}}
\newcommand{\darkred}[1]{\textcolor[rgb]{0.60,0.00,0.00}{#1}}
\newcommand{\brown}[1]{\textcolor[rgb]{0.50,0.30,0.00}{#1}}
\newcommand{\purple}[1]{\textcolor[rgb]{0.50,0.00,0.50}{#1}}% 为使用方便而编辑的新指令
%%%%%%%%%%%%%%%%%%%%%%%%%%%%%%%%%%%%%%%%%%%%%%%%%%%%%%%%
\usepackage{url}% 超链接
\usepackage{bm}% 加粗部分公式
\usepackage{multirow}
\usepackage{booktabs}
\usepackage{epstopdf}
\usepackage{epsfig}
\usepackage{longtable}% 长表格
\usepackage{supertabular}% 跨页表格
\usepackage{algorithm}
\usepackage{algorithmic}
\usepackage{changepage}% 换页
%%%%%%%%%%%%%%%%%%%%%%%%%%%%%%%%%%%%%%%%%%%%%%%%%%%%%%%%
\usepackage{enumerate}% 短编号
\usepackage{caption}% 设置标题
\captionsetup[figure]{name=\fontsize{10pt}{15pt}\selectfont Figure}% 设置图片编号头
\captionsetup[table]{name=\fontsize{10pt}{15pt}\selectfont Table}% 设置表格编号头
%%%%%%%%%%%%%%%%%%%%%%%%%%%%%%%%%%%%%%%%%%%%%%%%%%%%%%%%
\usepackage{indentfirst}% 中文首行缩进
\usepackage[left=2.50cm,right=2.50cm,top=2.80cm,bottom=2.50cm]{geometry}% 页边距设置
\renewcommand{\baselinestretch}{1.5}% 定义行间距(1.5)
%%%%%%%%%%%%%%%%%%%%%%%%%%%%%%%%%%%%%%%%%%%%%%%%%%%%%%%%
\usepackage{fancyhdr} %设置全文页眉、页脚的格式
\pagestyle{fancy}
\hypersetup{colorlinks=true,linkcolor=black}% 去除引用红框,改变颜色
%%%%%%%%%%%%%%%%%%%%%%%%%%%%%%%%%%%%%%%%%%%%%%%%%%%%%%%%
%自己添加的宏包
\usepackage{caption}%标题取消自动figure
\usepackage{amsmath}
\allowdisplaybreaks
%%%%%%%%%%%%%%%%%%%%%%%%%%%%%%%%%%%%%%%%%%%%%%%%%%%%%%%%
\begin{document}% 以下为正文内容
	\maketitle% 产生标题,没有它无法显示标题
	%%%%%%%%%%%%%%%%%%%%%%%%%%%%%%%%%%%%%%%%%%%%%%%%%%%%%%%%
	\lhead{}% 页眉左边设为空
	\chead{}% 页眉中间设为空
	\rhead{}% 页眉右边设为空
	\lfoot{}% 页脚左边设为空
	\cfoot{\thepage}% 页脚中间显示页码
	\rfoot{}% 页脚右边设为空
	%%%%%%%%%%%%%%%%%%%%%%%%%%%%%%%%%%%%%%%%%%%%%%%%%%%%%%%%
	
	\begin{center}% 居中处理
		{\textbf{Abstract}}% 英文摘要
	\end{center}
	\begin{adjustwidth}{1.06cm}{1.06cm}% 英文摘要内容
		\hspace{1.5em}
	\noindent Def: $f(x)$ is a function increasing monotonically in this interval $(0,+\infty)$, and
	\begin{equation*}
		\lim_{x\rightarrow +\infty} \frac{f\left( ax \right)}{f\left( x \right)}=1
    \end{equation*}
	\ \ \ \ \ \ proof: $\forall a>0$, we have
	\begin{equation*}
		\lim_{x\rightarrow+\infty}\frac{f(ax)}{f(x)}=1
	\end{equation*}
	\end{adjustwidth}
	\section*{Solution}
	\noindent We can use the pinch theorem. Obviously $\exists n\in \mathbb{Z},\  s.t.\ a\in[2^n,2^{n+1}]$. As a result
	\begin{align*}
		 \lim _{x\rightarrow +\infty}\frac{f\left( ax \right)}{f\left( x \right)} &=\left[ \lim _{x\rightarrow +\infty}\frac{f\left( 2^nx \right)}{f\left( x \right)},\lim _{x\rightarrow +\infty}\frac{f\left( 2^{n+1}x \right)}{f\left( x \right)} \right] \\\\
		 &=\left[ \lim _{x\rightarrow +\infty}\frac{f\left( 2^nx \right)}{f\left( 2^{n-1}x \right)}\frac{f\left( 2^{n-1}x \right)}{f\left( 2^{n-2}x \right)}\cdot \cdot \cdot \frac{f\left( 2x \right)}{f\left( x \right)},\lim _{x\rightarrow +\infty}\frac{f\left( 2^{n+1}x \right)}{f\left( 2^nx \right)}\frac{f\left( 2^nx \right)}{f\left( 2^{n-1}x \right)}\cdot \cdot \cdot \frac{f\left( 2x \right)}{f\left( x \right)} \right] \\\\
		 &=\left[ \lim _{x\rightarrow +\infty}\frac{f\left( 2^nx \right)}{f\left( 2^{n-1}x \right)}\lim _{x\rightarrow +\infty}\frac{f\left( 2^{n-1}x \right)}{f\left( 2^{n-2}x \right)}\cdot \cdot \cdot \lim _{x\rightarrow +\infty}\frac{f\left( 2x \right)}{f\left( x \right)},\right.\\\\
		 &\left.\ \ \ \ \ \lim _{x\rightarrow +\infty}\frac{f\left( 2^{n+1}x \right)}{f\left( 2^nx \right)}\lim _{x\rightarrow +\infty}\frac{f\left( 2^nx \right)}{f\left( 2^{n-1}x \right)}\cdot \cdot \cdot \lim _{x\rightarrow +\infty}\frac{f\left( 2x \right)}{f\left( x \right)}\right]\\\\
		 &\rightarrow \left\{ 1 \right\}
	\end{align*}
	
	\newpage% 从新的一页继续

	
\end{document}% 结束文档编辑,后面写啥都编译不出来