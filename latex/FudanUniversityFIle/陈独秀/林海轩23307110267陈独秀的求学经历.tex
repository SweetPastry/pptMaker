\documentclass[10pt,a4paper]{beamer} %ppt模式

\usepackage{ctex}

\usepackage{beamerthemesplit} % 加载主题宏包
\usetheme{Warsaw} % 选用该主题

\usepackage[T1]{fontenc}

%插入图片, 写公式, 画表格等
\usepackage{subfig}
\usepackage{amssymb,amsmath,mathtools}
\usepackage{amsfonts,booktabs}
\usepackage{lmodern,textcomp}
\usepackage{color}
\usepackage{tikz}
\usepackage[utf8]{inputenc}
\usepackage{natbib}
\usepackage{multicol}
\usepackage{graphicx}
\setbeamertemplate{footline}{}%清除底部信息,更美观
\setbeamertemplate{headline}{
	\begin{beamercolorbox}[wd=\paperwidth,ht=4ex,dp=3ex]{section in head/foot}
		\hfill\insertsection\hspace*{2ex}
	\end{beamercolorbox}
}
\begin{document}
	
	%基本信息
	\title{陈独秀的求学经历}
	\subtitle{近纲课}
	\author{林海轩}
	\institute{复旦大学物理学系}
	\date{}
	
	%生成标题页
	\begin{frame}
		\titlepage
	\end{frame}
	
	%目录页,代码无需改动
	\begin{frame}
		\tableofcontents
	\end{frame}
	
	
	%%%%%%%%%%%%%%%%%%%%%%%%%%%%%%%%%%%%%%%%%%%%%%%%%%%%%%%%%%%%%%%%%%%%%%%%%%%%%%%%%%%%%%%%%%%%%%%
	\section{杭州求中西是书院学习}
	\begin{frame}
		陈独秀于1897年进入杭州中西求是书院,这是他首次接触到近代西方思想文化。在书院,他学习了西方科学知识,也受到了西方民主思想的熏陶。这段学习经历对陈独秀的世界观和人生观产生了深远的影响,为他日后成为革命家和思想家打下了坚实的基础。据史料记载,陈独秀在书院期间表现出色,成绩优异,同时也积极参与书院的各种活动,锻炼了自己的组织和领导能力。
	\end{frame}
	
	\section{上海协助章士钊主编《国民日报》}
	\begin{frame}
		1903年,陈独秀前往上海,协助章士钊主编《国民日报》。这是他首次涉足新闻传媒领域,也是他宣传革命思想、推动社会进步的重要平台。在《国民日报》上,陈独秀发表了大量政论文章,抨击封建专制制度,呼吁民主自由,为当时的革命运动提供了舆论支持。这段经历不仅展现了陈独秀敏锐的政治洞察力,也彰显了他坚定的革命信念。
	\end{frame}
	
	\section{芜湖创办《安徽俗话报》}
	\begin{frame}
		1904年初,陈独秀在芜湖创办了《安徽俗话报》。这份报纸以通俗易懂的语言向广大民众普及科学知识和革命思想,成为当时安徽地区重要的革命宣传阵地。陈独秀通过《安徽俗话报》,广泛传播了民主革命的思想,激发了民众的爱国热情,推动了安徽地区的革命运动。这份报纸的创办,不仅体现了陈独秀深厚的爱国情怀,也展现了他卓越的组织和宣传能力。
	\end{frame}
	
	\section{组织反清秘密革命组织岳王会}
	\begin{frame}
		1905年,陈独秀组织反清秘密革命组织岳王会,并担任总会长。这个组织以推翻清朝封建统治、建立民主共和国为目标,进行了一系列秘密的革命活动。陈独秀作为总会长,领导岳王会成员开展革命宣传、组织武装起义等工作,为当时的革命运动做出了重要贡献。这段经历充分展示了陈独秀的革命勇气和坚定信念,也奠定了他在革命运动中的重要地位。
	\end{frame}
	
	\section{赴日本留学}
	\begin{frame}
		1907年,陈独秀前往日本留学,进入东京正则英语学校学习。随后,他又转入早稻田大学深造。在日本的留学生活中,陈独秀深入学习了西方的政治、经济、文化等方面的知识,进一步拓宽了视野,增强了自身的素养。这段留学经历对陈独秀的思想和行动产生了深远的影响,为他日后成为一位杰出的革命家和思想家奠定了坚实的基础。
	\end{frame}
	
	\section{浙江陆军学堂任教}
	\begin{frame}
		1909年冬,陈独秀离开日本,前往浙江陆军学堂任教。在这所军事学校中,他不仅教授学生军事知识,还积极传播革命思想,鼓励学生投身革命运动。他的教学活动受到了学生们的热烈欢迎,也为当时的革命运动培养了一批骨干力量。这段任教经历再次证明了陈独秀在革命宣传和组织方面的卓越才能。
	\end{frame}
	
	\section{担任安徽省都督府秘书长}
	\begin{frame}
		辛亥革命爆发后,陈独秀积极投身革命斗争,并于1911年担任安徽省都督府秘书长。在这个职位上,他协助都督处理政务,参与制定革命政策,为安徽地区的革命运动提供了有力的支持。他的工作得到了革命政府的认可和赞誉,也进一步提升了他在革命运动中的地位和影响力。
	\end{frame}
	
	\section{创办并主编《青年》杂志}
	\begin{frame}
		1915年9月,陈独秀在上海创办并主编了《青年》杂志(一年后改名为《新青年》)。这份杂志以提倡民主、科学为主要宗旨,成为新文化运动的重要阵地。陈独秀通过《新青年》发表了大量文章,抨击封建礼教和旧道德,提倡新思想和新文化,激发了一代青年人的觉醒和进步。这份杂志的创办和主编经历,不仅展示了陈独秀在新文化运动中的领导地位,也奠定了他在中国现代思想史上的重要地位。
	\end{frame}
	
	\section{受聘为北京大学文科学长}
	\begin{frame}
		1917年初,陈独秀受聘为北京大学文科学长。在这个职位上,他积极推行教育改革,提倡学术自由和思想解放,吸引了一批优秀的学者和学生加入北京大学。他的改革措施不仅提升了北京大学的学术水平和社会影响力,也为中国现代高等教育的发展做出了重要贡献。同时,陈独秀在北大期间继续发表文章和演讲,推动新文化运动的发展,影响了一代又一代的知识分子。
	\end{frame}
	\section{与李大钊等创办《每周评论》}
	\begin{frame}
		1918年12月,陈独秀与李大钊等人一起创办了《每周评论》。这份杂志以评论时事、传播新思想为主要内容,成为当时社会舆论的重要平台。陈独秀通过《每周评论》发表了大量文章和评论,对当时的政治、经济、文化等方面进行了深入的剖析和批判。这份杂志的创办不仅进一步扩大了陈独秀的社会影响力,也推动了中国社会的进步和发展。
	\end{frame}
	
	%%%%%%%%%%%%%%%%%%%%%%%%%%%%%%%%%%%%%%%%%%%%%%%%%%%%%%%%%%%%%%%%%%%%%%%%%%%%%%%%%%%%%%%%%%%%%%%	
	
	%参考文献(非调用.bib文件,而是手动输入)
	\appendix
	\begin{frame}{参考文献}
		\begin{thebibliography}{99} % 最大可能的参考文献数目,可以根据实际情况调整
			\bibitem[Author et al., 2021]{ref1}
			Dewey. Art as Experience[J]. 高等教育出版社, 11.4(2022):1-4.
			\newblock Title of the first reference.
			\newblock \emph{Journal Name}, \emph{Volume}(Issue), PageRange.
			
			\bibitem[Another Author, 2022]{ref2}
			Plato. Utopia[M]. 高等教育出版社, 01(2333):-2-4.
			\newblock Title of the second reference.
			\newblock \emph{Another Journal}, \emph{Volume}(Issue), PageRange.
		\end{thebibliography}
		\bibliographystyle{plain}
		\bibliography{ypzz}
	\end{frame}
	
	\begin{frame}[plain,c]
		\begin{center}
			\Huge 感谢聆听 !
		\end{center}
	\end{frame}
	
\end{document}