%-*- coding: UTF-8 -*-
% paper.tex
\documentclass[twocolumn]{ctexart}
\usepackage{lipsum,mwe,cuted}
\usepackage{float}%%%%提供浮动体的[H]选项,进而取消浮动
\usepackage{caption}%%提供\captionof命令
\usepackage{geometry}

\pagestyle{plain}
% \geometry{a4paper,left=2cm,right=2cm,top=2cm,bottom=2cm}

\geometry{a4paper,scale=0.8}

\title{对杜威“感觉是直接呈现在经验\\之中不证自明的意义”观点的评说}
\author{ 林海轩\ 学号:23307110267 \\\ (复旦大学 \ 物理学系)}
\date{\vspace{-2em}}

\stripsep 8pt
\pagestyle{plain}
\newtheorem{thm}{定理}
\bibliographystyle{unsrt}
\CTEXsetup[format={\Large\bfseries}]{section}
\newcommand{\upcite}[1]{\textsuperscript{\textsuperscript{\cite{#1}}}}
\usepackage{amsthm}%引理无序号,\newtheorem*{theorem}{定理}这样定义的定理是没有编号的
\usepackage{indentfirst}

\begin{document}
	
	\maketitle
	
	\begin{strip}
		
		\noindent  \textbf{摘要} .\ 基于杜威《艺术即经验》中的观点完成第三次讨论课的个人论文,以下是助教给出的题目:
		\newtheorem*{lemma}{问题}
		\begin{lemma}在《艺术即经验》第二章第5段,杜威说“感觉是直接呈现在经验之中不证自明的意义”,请对此试作评说。
		\end{lemma}
		
		\leavevmode\\
	\end{strip}
	
	\section*{1  引言}
	杜威强调感觉是直接呈现在经验中不证自明的意义,这一观点突显了感觉在我们对世界的认知和艺术体验中的基础作用。然而,这一观点也引发了深刻的思考和潜在的争议。本文将对杜威的观点展开深入讨论,探讨感觉的直观性、复杂性以及与认知、文化的关系,同时结合生活例子和文学经典,对这一观点进行拓展。
	
	\section*{2  正文}
	
	首先,值得肯定的是,感觉确实在我们日常经验中起着至关重要的作用。无论是看到一幅画、听到一首音乐,还是触摸到某物,感觉是我们获取信息、理解世界的最直接途径之一。杜威的观点突显了感觉的即时性和不需要经过推理的特性,这与艺术创作和欣赏过程中强调的直觉体验紧密相连。
	
	我们可以从日常生活中找到许多例子来支持感觉的直观性。比如,当我们看到一片蓝天或感受到阳光的温暖时,这些感觉无需经过深思熟虑,直接而即时地构成了我们的经验。在艺术中,画家通过色彩的运用,音乐家通过音符的组合,都在追求观众在感觉上的直接共鸣。这种直观性使得感觉成为艺术创作和欣赏的核心元素。
	
	然而,我们也需要注意到,感觉并非是一个孤立的元素,而是与认知、文化、个体经历等因素交织在一起的。在一定程度上,感觉的意义可能并非完全不证自明,而可能受到认知框架、文化传统和个体经验的影响。例如,在不同的文化背景中,对于某种颜色或图像的感觉可能会产生不同的解释和情感反应。这一点使得感觉的直观性和不证自明性变得相对而言,而非绝对。
	
	从认知科学的角度来看,感觉与认知过程之间存在着复杂的相互作用。感觉是认知的一部分,但认知也可以调节感觉的解释和理解。这种相互作用表明,感觉不仅仅是被动接受的信息,而可能受到个体思维和认知架构的主动塑造。
	
	借用康德的观点,他在《纯粹理性批判》中强调经验的直观性和先验性。康德认为经验始于感觉,而这种感觉又是直接而无需推理的。然而,康德也指出,认知的先验结构对感觉的解释起着至关重要的作用。在文学作品中,我们常常看到作家通过描绘感官细节,以唤起读者的感觉体验。但这些感觉又往往受到文学结构和主题的塑造,由此可以看出感觉与认知之间的错综复杂的关系。
	
	在艺术领域,尤其是在抽象艺术或实验性艺术的情境下,感觉的解释可能更加开放和主观。观众在欣赏艺术品时可能经历着不同的感觉体验,而这些体验往往是复杂多样的,涉及到情感、记忆和审美情趣。这种多样性可能表明感觉在艺术中的意义并非单一而固定的,而是具有灵活性和个体化的特点。
	
	从进化的角度看,感觉系统的发展和演化是为了帮助生物适应环境、获取生存所需的信息。这一生存需要的直觉性可能构成了感觉不证自明性的一部分。比如,当我们感知到某物的危险性,身体会自发地产生紧张和警觉,这种感觉有助于我们的自我保护。这一进化的视角提供了感觉直观性的一种基础解释,即感觉的直观性在进化中可能具有生存的适应性。
	
	
	
	
	\section*{3 总结}
	
	综合而言,杜威关于感觉在经验中不证自明的意义的观点提供了一个重要的思考角度。感觉的直观性在艺术和日常生活中都具有显著作用,但我们也需要认识到感觉与认知、文化之间的复杂相互作用。生活例子和文学经典的引证为我们提供了更具体的论据,进一步拓展了对这一观点的理解。感觉的多样性和进化的角度为我们提供了更全面、多层次的视角,使我们更好地理解感觉在塑造我们对世界的认知和艺术体验中的作用。这一探讨不仅有助于深化对感觉的理解,也为艺术哲学和认知科学的交叉领域提供了丰富的思考空间。
	
	
	
	
	
	\begin{thebibliography}{99}% 参考文献,{}内表示序号最大位数(两位)
		\bibitem{ref1}Dewey. Art as Experience[J]. 高等教育出版社,11.4(2022):1-4.
		\bibitem{ref2}Plato. Utopia[M]. 高等教育出版社, 01(2333):-2-4.
	\end{thebibliography}
\end{document}