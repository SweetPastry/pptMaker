%-*- coding: UTF-8 -*-
% paper.tex
\documentclass[twocolumn]{ctexart}
\usepackage{lipsum,mwe,cuted}
\usepackage{float}%%%%提供浮动体的[H]选项,进而取消浮动
\usepackage{caption}%%提供\captionof命令
\usepackage{geometry}

\pagestyle{plain}
% \geometry{a4paper,left=2cm,right=2cm,top=2cm,bottom=2cm}

\geometry{a4paper,scale=0.8}

\title{关于杜威“感觉是直接呈现在经验\\之中不证自明的意义”观点的评说}
\author{ 林海轩\ 学号:23307110267 \\\ (复旦大学 \ 物理学系)}
\date{\vspace{-2em}}

\stripsep 8pt
\pagestyle{plain}
\newtheorem{thm}{定理}
\bibliographystyle{unsrt}
\CTEXsetup[format={\Large\bfseries}]{section}
\newcommand{\upcite}[1]{\textsuperscript{\textsuperscript{\cite{#1}}}}
\usepackage{amsthm}%引理无序号,\newtheorem*{theorem}{定理}这样定义的定理是没有编号的
\usepackage{indentfirst}

\begin{document}
	
	\maketitle
	
	\begin{strip}
		
		\noindent  \textbf{摘要} .\ 基于杜威《艺术即经验》中的观点完成第三次讨论课的个人论文,以下是助教给出的题目:
		\newtheorem*{lemma}{问题}
		\begin{lemma}
			在《艺术即经验》第二章第5段,杜威说“感觉是直接呈现在经验之中不证自明的意义”,请对此试作评说。
		\end{lemma}
		
		\leavevmode\\
	\end{strip}
	
	
	\section*{1  正文}
	
		杜威正试图给“感觉”做一个定义,而这种定义是基于对经验的理解的,为了深入解析这句话的内涵,我们需要先理解“经验”的概念,并尝试构建“经验”与“感觉”的联系,我将从一下五个方面拆解这个观点。
	
		第一点,感觉的直觉性。杜威在这里强调感觉是一种直接的、非中介的经验元素,通过感官直接与环境互动,而不需要复杂的思考或分析。
		
		第二点,感觉的自明性。在杜威的陈述中,他使用了“不证自明”这个词汇。任何智慧的建立都需要最根本的判据,这种判据叫做“公理”或者“假设”,显然它们必须是最具有说服力的,为了达成这一目标,它们必须来源于人类的直觉经验。可以说,感觉的直觉性是智慧的发端,体现了其“不证自明”的特点。自明性可能指的是感觉作为一种基本经验元素存在时,无需进一步解释或证明其存在的合理性。感觉是我们存在的一部分,这一存在是如此基本和明显,以至于我们无需对其进行论证。
	
	
	
	\section*{2 总结}
		杜威在这句话中表达了他的实用主义美学观,即认为艺术是人与环境相互作用的经验的表现,而感觉是经验的基本要素,是人对外界刺激的直接反应。感觉不需要通过任何理性的推理或逻辑的分析来证明它的意义,它就是经验的意义,它就是艺术的意义。杜威反对把艺术与日常生活经验分离,把艺术看作一种超越的、理想的、精神的现象,他认为这样会剥夺艺术的生命力和活力,使艺术变成一种“博物馆艺术观”。他主张把艺术还原到人的具体经验之中,把感觉作为经验的核心,把艺术作为感觉的表现。
	
		我认为,杜威的这种观点有一定的合理性,它强调了艺术与生活的联系,强调了感觉在艺术中的重要作用,强调了艺术的动态性和创造性。但是,它也有一些不足之处,比如忽视了理性在艺术中的作用,忽视了艺术的规律性和普遍性,忽视了艺术与自然和社会的关系。因此,杜威的美学观需要进一步的完善和发展,不能把它当作艺术的唯一标准和准则。
	
	
	
	
	
	\begin{thebibliography}{99}% 参考文献,{}内表示序号最大位数(两位)
		\bibitem{ref1}Dewey. Art as Experience[J]. 高等教育出版社,11.4(2022):1-4.
		\bibitem{ref2}Plato. Utopia[M]. 高等教育出版社, 01(2333):-2-4.
	\end{thebibliography}
\end{document}