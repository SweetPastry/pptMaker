\documentclass[10pt,a4paper]{beamer} %ppt模式

\usepackage{ctex}

\usepackage{beamerthemesplit} % 加载主题宏包
\usetheme{Warsaw} % 选用该主题

\usepackage[T1]{fontenc}

%插入图片, 写公式, 画表格等
\usepackage{subfig}
\usepackage{amssymb,amsmath,mathtools}
\usepackage{amsfonts,booktabs}
\usepackage{lmodern,textcomp}
\usepackage{color}
\usepackage{tikz}
\usepackage[utf8]{inputenc}
\usepackage{natbib}
\usepackage{multicol}
\usepackage{graphicx}
\setbeamertemplate{footline}{}%清除底部信息,更美观

\begin{document}
	
	%基本信息
	\title{李鸿章掌管的政务}
	\subtitle{近纲ppt}
	\author{林海轩}
	\institute{复旦大学物理学系}
	\date{}
	
	%生成标题页
	\begin{frame}
		\titlepage
	\end{frame}
	
	%目录页,代码无需改动
	\begin{frame}
		\tableofcontents
	\end{frame}
	
	
%%%%%%%%%%%%%%%%%%%%%%%%%%%%%%%%%%%%%%%%%%%%%%%%%%%%%%%%%%%%%%%%%%%%%%%%%%%%%%%%%%%%%%%%%%%%%%%
	\section{雇佣外军镇压农民起义}
	\begin{frame}
		李鸿章,字子黻,号少荃,安徽合肥人,道光三年正月初五日(1823年2月5日)出生于封建官僚地主家庭。父李文安曾官刑部郎中,记名御史。李鸿章于道光二十四年(1844)考中举人。翌年,通过父亲与曾国藩同年的关系,师事曾氏,“日夕过从,讲求义理经世之学”。过了两年,李鸿章考中进士。道光三十年(1850),授翰林院编修。太平天国革命爆发后,李鸿章曾多次率领团勇协助安徽地方官员对抗太平军。咸丰八年(1852),李鸿章奔江西晋谒曾国藩,入其营幕。次年十月,被任命为福建延建邵遗缺道,未曾到任。在曾幕期间,李鸿章郁郁不得志,师徒之间微有龃龉,曾一度拂袖他去,“闲居一年”。咸丰十一年(1861)秋,湘军攻陷安庆,他又回到老师的营幕,这时,曾国藩对他另眼相看,举凡“军国要务,皆与筹商”。李鸿章受到重用,从此成为曾国藩镇压太平天国农民起义的左右手。
	\end{frame}
	\begin{frame}
		丰十年(1860)太平军进军苏、杭,威胁上海。李鸿章奉命回安徽招募淮勇,皖籍地主武装张树声、周盛波、潘鼎新、刘铭传等人纷纷响应,不久,一支拥有6500余人的淮军正式编成。淮军建立之初,“营伍之法,器械之用,薪粮之数,悉仿湘军章程”。曾国藩调拨湘军数营并入该军,又以湘军悍将程学启、郭松林等人充当军中头目。同治元年(1862)夏,李鸿章统率淮军由安庆乘英轮来到沪上,旋奉命署江苏巡抚。李鸿章从曾国藩手下的一个幕僚,一变成为清朝统治集团重要的一员,并在军事上取得独当一面的指挥权。
	\end{frame}
	\begin{frame}
		淮军抵沪后,曾国藩谆谆告诫李鸿章对待侵略者要恪守“忠信笃敬”信条。李鸿章又看到外国军队拥有新式枪炮,叹为“神技”,“日诫谕将士虚心忍辱,学得西人一二秘法”①。他一面以重金聘请洋教官到各营教习;一面向侵略者购买新式武器。此外,他还“全神笼络”华尔,企图通过他向外国聘请“铁匠制炸弹,代购洋枪”②,并以他“一人之心”去“联络各国之好”③。
		
		为了认真训练淮军,并与外国侵略者协同作战,李鸿章与何伯以及英国陆军提督士迪佛立订立会商攻打太平军协议,规定:(1)李鸿章派出淮军6000人与侵略军“共维大局”,其中3000人进驻南桥,3000人由英国军官负责严加训练;(2)作战中夺取的军需品应归淮军所有;(3)双方军事调动必须互相通知,并互相供给军事情报。这样,李鸿章与资本主义侵略者正式建立了共同镇压太平军的军事合作。
		
		同治二年(1863)春,李鸿章与士迪佛立订立《会同管带常胜军条约十六款》,议明:(1)全军编制3000人,暂以奥伦为统领,清军以副将李恒嵩会同管带;(2)常胜军及统带官由抚台(李鸿章)指挥;(3)军费开支由海关收入供应,军火由李鸿章购买拨给。常胜军整顿后不久,戈登取代奥伦为统领,李鸿章很快“成为戈登的好友和赞助他的人”④。
	\end{frame}
	\begin{frame}
		
		
		常胜军整顿后,的确帮了李鸿章的大忙。同治二年三月(1863年5月),太仓陷落,太平军被“追斩殆尽”①,李鸿章赞扬戈登“坚忍镇定”。不久,昆山失守,太平军7000余人被俘,3万人遭杀害,李鸿章对戈登的“奋勇”表示“殊堪嘉尚”②。十月(12月),苏州被攻占,李鸿章率军大肆抢劫财物,并发给常胜军7万元,后奉旨另犒赏戈登银1万两,以资鼓励。次年四月,攻陷常州,常胜军宣告解散,留洋枪队300人、炮队600人并入淮军。攻陷苏、常后,清军加紧进攻天京。同治三年六月十六日(1864年7月19日),攻陷天京。李鸿章又调派淮军追杀太平军余部。由于平定“粤寇”有功,清廷赐封李鸿章“一等肃毅伯”。他的淮军由原来的6000余人逐渐扩充到六七万人,逐步取代湘军成为清廷所倚重的一支凶恶军队。在劫掠人民财富方面,李鸿章也并不比曾国藩兄弟逊色,李家兄弟数人在绞杀太平军期间,就兼并土地约60万亩。
		
		太平天国起义被扼杀后,清廷集中兵力围剿捻军。同治五年(1866)秋,李鸿章吸取了曾国藩失败的教训,认为要扑灭“倏忽无定”的捻军,必须增添马队,马步配合,左右夹击,前后堵截,才能扭转尾随追逐,劳而无功的局面。在战略上,他坚守曾国藩“画河圈地”的主张,实行“扼地兜剿”,驱逐捻军到“山深水复之处”③,重重围困,加以歼灭。与此同时,还“阴招其饥疲裹胁之众,使其内乱残杀”①。同治六年十二月(1868年1月),赖文光被俘就义,东捻军为李鸿章所扑灭。同治七年正月(1868年2月),西捻军进入直隶,“京师大震”,清廷急调李鸿章、左宗棠进行镇压。七月(8月),西捻军进入山东境内黄河、运河和徒骇河之间的狭窄地带,被清军围困,无法突围失败。
		
		李鸿章在镇压捻军中立下汗马功劳,于同治六年(1867)被清廷实授为湖广总督,西捻军覆灭,又赏加“太子太保衔”,成为当时一个握有军政实权的封疆大吏,也是为清廷所倚重的一个“中兴名臣”。
	\end{frame}
	\section{洋务运动}
	\begin{frame}
		1860年《北京条约》之后,列强暂停了对中国的军事侵略,转向大力开拓对华贸易。中外贸易额,由四五十年代的每年一千万至三四千万海关两,迅速上升到1864年的1.05亿两。中国很快由出超国变为入超国。1869年,苏伊士运河开通了,欧洲到中国的航线大为缩短,更多的西洋商品涌入中国。随大量西方工业品而来的新一轮“西学东渐”浪潮,促使晚清中国的洋务运动发生渐进的质变。   
		
		1870年,大清帝国的京畿重地天津,爆发了一场中国近代史上的大事。法国领事丰大业(M. Fontanier)枪击中国民众,愤怒的人群将其打死,并焚烧了法国天主教堂。法国联合英、美、德、意等六国调集军舰到天津、烟台,要求清廷严惩“肇事者”。当局把两江总督曾国藩从南京调去平息事端。曾深感“教案皆百姓积不能平所致”, “……此次天津之案……唯有委曲求全之一法”。于是,他以16个中国人头落地换取西方七国息怒。这件事情使曾国藩受到国内广泛抨击,甚至被骂为“卖国贼”,他自己也深感内疚。清廷随即将其调离京畿,转回两江总督。1872年初,晚清“中兴第一名臣”曾国藩,便在郁闷中病逝。
		
		曾国藩的直隶总督兼北洋大臣职位,是由时任两江总督的李鸿章接任的。李鸿章(1823~1901年,字少荃)是安徽庐县(今合肥)人,也是从曾国藩的幕僚中锻炼出来的封疆大吏。
		
		从1861年美国在华设立旗昌轮船公司开始,航行于中国沿海及内河的各国商船逐渐增多。在一些河段,甚至90%以上的航运被外商控制。机动船严重冲击旧式航运业,中国船户纷纷破产。并且,“洋船往来长江,实获厚利,喧宾夺主,害不独商。”
		
		为“庶使我内江外海之利,不致为洋人占尽”,1872年底,李鸿章奏请筹建轮船招商局,得到慈禧太后的“恩准”,并同意由户部借制钱20万串(合白银10万两),“以作设局商本,而示信于众商”。同年12月,轮船招商局在上海成立。该局主要经营长江、海洋航运,“揽载为第一义,运漕为第二义”。其中,漕运业务由朝廷划拨;主要揽载旅客、进出口成品和原料运输。到1874年,该局召集到的总股本为47万两白银。
		
		轮船招商局成立不久,各国在华轮船公司随即展开恶性竞争。在华赚取了十多年暴利的美国旗昌轮船公司轮船老化,缺乏竞争力,地盘日益减小。到后来,其独享的汉口、九江之利,也因招商局“江宽、江永两船到”而“气夺”。加上美国内战结束,国内市场诱惑力巨大,旗昌公司决定出售。1877年,在198万两官银的支持下,招商局投资222万两收购了旗昌所属旧船和设备。该局船只数量从12艘增加到33艘,吨位突破400万吨,占当时在各通商口岸进出的中外轮船吨位总数的36.7%。
	\end{frame}
	\begin{frame}
		招商局收购了旗昌公司后,英国太古、怡和等“洋商忌之益深,极力跌价倾轧”。清廷认为与洋船竞争“关系商务,不可半途而废,致为外人耻笑,并堕其得专中国利权之计”,加大对船局的扶助:官方贷款“分年还本,缓缴利息”;“自光绪四年(1878年)起,苏浙海运漕米必须照四、五成一律加拨,不准再有短少”;“沿江沿海各省遇有海运官物,应需轮船装运者统归局船照章承运”;“准令华商轮船在沿江沿海及内河不通商口岸自行贸易。”
		
		在降价竞争中损失惨重的太古、怡和公司,见无法压垮招商局,不得不在1878年与之达成妥协,签订了“齐价合同”。
		
		1882年前后和1890年,怡和、太古公司又展开恶性竞争,甚至以三四成乃至一成、五厘的价格抢客户,使得招商局“股价日跌”、赢利锐减。招商局“请将采运局平粜免税之米、援官物例归局专装以抵御之”,使得疲惫不堪的外国公司不得不再次妥协,分别于1883年、1893年两次签订“齐价合同”。
		
		对旗昌公司的兼并和几次“齐价合同”,在当时被认为是轮船招商局、李鸿章、洋务派和大清帝国的重大胜利,一些洋人也大加奉承。李鸿章不无得意地说:赴烟台时碰到法俄公使,他们都“称招商局办理深合机宜,为中国必不可少之举。任事诸人,措置亦甚得当。则此局之设,争利者深忌之,其不争利者未尝不服也”。
		
		1877年,李鸿章还在天津筹建了开平矿务局。该局早期招股不顺利,股本主要是直接的政府投资和轮船招商局的投资。出煤之后,李鸿章奏报:从此中国兵商轮船及机器制造各局用煤,不致远购于外洋。输入到天津的洋煤,从1882年的5400多吨,下降为1886年的301吨。
		
		为解决开平煤矿的运输问题,矿局又于1881年投资兴建了全长11公里的唐(山)胥(各庄)轻便铁路(时称快车路)。1886年成立开平铁路公司(后改组为中国铁路公司),将唐胥铁路逐渐延伸到芦台、天津。
	\end{frame}
	\begin{frame}
		1878年,李鸿章还筹建了上海机器织布局。在天津商界名人杨宗濂兄弟的经营下,该局1891年“每日用五百两,获利约五百两;每月可得一万二千利”。
		
		此外,这个时期著名的洋务企业还有几个,大部分为李鸿章集团所主导创办。
		
		1880年,左宗棠在兰州创建机器织呢厂,主要生产军用布料。
		
		1880年10月,李鸿章在天津设立由盛宣怀负责的电报总局和学堂。1882年4月改为官督商办。到甲午战争前,东北、辽东半岛、长江上下游甚至陕甘的主要城市和国防要塞,大都架通了电报线。
		
		1881年,李鸿章奏请开办承德平泉州铜矿制造子弹。
		
		1887年,李鸿章和黑龙江将军恭镗筹建了漠河金矿。【李鸿章创办了不少近现代企业,这方面功不可没。】
		
		1889年,张之洞在广州筹建了中国第一个近代化钢铁厂,后迁往武汉,更名汉阳铁厂。该厂于甲午战争之前投产,主要生产新式枪炮和铁轨。
		
		1890年至1893年,张之洞先后筹建了湖北织布局、纺纱厂、缫丝局。
		
		此外,还有贵州青 铁矿、山东淄川铅矿、云南铜矿、基隆煤矿等等。
		
		在以上企业之外,晚期洋务运动还继1862年的京师同文馆、1863年的上海方言馆、1866年的福州船政学堂等学堂之后,创设了一系列西学学堂:1880年的天津北洋水师学堂,1885年的天津武备学堂,1886年的广东陆师学堂,1887年的广东水师学堂,1890年的南京水师学堂,1892年的湖北铁政局附属化学堂、矿学堂,1893年的天津医学堂,1894年的湖北自强学堂等等。
		
		这个时期,还延续并拓展了容闳、丁日昌、曾国藩等倡议并奏请实施的向西方派遣留学生的计划,其中有著名的稚童留美活动。在从1872年到1875年的这个活动中,清政府先后向美国选派了120名幼童。他们在学会了美国语言之后,三分之一专修法律,其余三分之二学习航运、路矿、电报、医学、教育等等。这些“新式人才”,虽然后因美国社会的排华浪潮而回国,却在尔后的洋务运动中发挥了不可忽视的作用。【这批留美学生对中美关系的影响力深远,延伸到国民党时期的中美合作。】
		
		从历史的角度看,在与同光兴盛密不可分的洋务运动中,成果最大、对甲午战争影响最直接、最深重的,无疑是李鸿章一手创办、苦心经营的北洋海军。
	\end{frame}
	\section{外交}
	\begin{frame}
		1901年11月7日,大清王朝显赫一时的封疆大吏李鸿章终于不堪沉疴而溘离那个风雨动荡的年代。自1870年始李鸿章担任直隶总督,北洋通商大臣和文华殿大学士后,操纵晚清对外军事、外交和经济大政达三十年之久,正如时人所说,李鸿章“坐镇北洋,遥指朝政,凡内政外交,枢府常依为主,在汉臣中权势为最巨。”①但人们也不无遗憾地看到在其一手操办的对外交涉,诸如天津教案、中日修好条规、中法新约、马关条约、中俄密约、……无一胜数,堪称是民族之不幸。因此有人认为,就“李鸿章的个人历史,几乎就是近代中国国力衰败的历史,是近代中国沦为外国列强半殖民地的历史”。
	\end{frame}
	\begin{frame}
		在19世纪60年代的中国,开始经历了一次重大的外交军事和工业生产方式的变革。这里首先是由于西方列强在侵略中国的活动中采取了“合作政策”。它们为了巩固和扩大自第一次鸦片战争以来所签订的一系列不平等条约中所攫取的利益,在有关侵华的重大问题上彼此进行“协商与合作”,以达到共同的侵略目的。在列强虎视眈眈之下,清政府亦强烈的感觉到“彼族深阴狡黠,遇事矫执,或条约中本系明晰,而彼必曲伸其说,或条约中未臻妥善,而彼必据以为词,极其坚韧性成,得步进步,不独于约内所已载者难稍更动,且思于约外未载者更为增添。”② 然而,此时清政府已无力抗拒列强的要求,只能是卑躬屈膝,对外一味采取妥协退让的方针。在这种所谓的“中外和好”的“和局”面前,经历了绞杀太平天国及捻军起义的高官大吏,诸如曾国藩、李鸿章、左宗棠之流亲眼看到了西方侵略者船坚炮利的“长技”,从而预感到一种潜在的长远威胁。1860年《北京条约》签署后,曾国藩就提出:“此次款议虽成,中国岂可一日忘备?……目前资夷力以助剿济远,得纾一时之忧,将来师夷智以造炮制船,尤可期永远之利。”③ 1861年3月,曾国藩又再次强调购买外洋船炮乃是“今日救时之第一要务”。指出“轮船之速,洋炮之远,在英法则夸其独有,在中华则罕于所见”,进而主张应“广访募覃思之士,智巧之匠,始而演之,继而试造,不过一二年,火轮船必为中外官民通行之物,可以剿发逆,可以存远略。”④ 而李鸿章则更因在1862年率领淮军到达上海与英法侵略军和华尔的“常胜军”向太平军进攻时亲眼看到外国军队的“落地开花炸弹”而赞不绝口,视为“神技”。他为此也曾给曾国藩写信哀叹到“深以中国军器远逊外洋为耻。”⑤ 转而愤言到:“外国利器强兵,百倍中国,内则狎处辇轂之下,外则布满江湖之间,”“外国猖獗至此,不亟亟焉求富强,中国将何以自立耶?”⑥ 在这里,他们所看到的只是中国在武器装备和科学技术方面大大地落后于西方,故转而欲继承早年鸦片战争时期“经世派”代表人物魏源提出的“师夷长技”的思想,并且极力把这一思想主张付诸实践。他们“师夷长技”的目的,一是为平定国内日益尖锐的阶级矛盾,维护清朝的统治,二是要在与外国侵略者保持“和好”的条件下,徐图自强,免遭沦胥。关于这一点李鸿章有过告白“目前之患在内寇,长久之患在西人”,“似当委曲周旋,但求外敦和好,内要自强。”⑦ 而在同治三年(一八六四年)他又曾写信给恭亲王奕訢和文祥说:
	\end{frame}
	\begin{frame}
		“鸿章窃以为天下事穷则变,变则通。中国士大夫沉浸于章句小楷之积习,武夫悍卒又多粗蠢而不加细心,以致用非所学,学非所用。无事则斥外国之利器为奇技淫巧,以为不必学,有事则惊外国之利器为变怪神奇,以为不能学。不知洋人视火器为身心性命之学者已数百年。一旦豁然贯通,参阴阳而配造化,实有指挥如意,从心所欲之快。……前者英法各国,以日本为外府,肆意诛求。日本君臣发愤为雄,选宗室及大臣子弟之聪秀者,往西国制器厂师习各艺,又购制器之器,在本国制习。现在已能驾驶轮船,造放炸炮。去年英人虚声恫愒,以兵临之。然英人所恃而为攻战之利者,彼已分擅其长,用是凝然不动,而英人固无如之何也。夫今之日本即明之倭寇也,距西国远而距中国近。我有以自立,则将附丽于我,窥伺西人之短长;我无以自强,则并效尤于彼,分西人之利薮。日本以海外区区小国,尚能及时改辙,知所取法。然则我中国深维穷极而通之故,夫亦可以皇然变计矣。……杜挚有言曰:利不百,不变法。功不十,不易器。苏子瞻曰言之于无事之时,足以为名,而恒苦于不信;言之于有事之时,足以见信,而已苦于无及。鸿章以为中国欲自强则莫如学习外国利器。欲学习外国利器,则莫如觅制器之器,师其法而不必尽用其人。欲觅制器之器,与制器之人,则我专设一科取士,士终身悬以为富贵功名之鹄,则业可成。业可精,而才亦可集”。⑧ 应当说,这封信是中国十九世纪最大的政治家,最具历史价值的一篇文章。正如后来的学者在分析中所认为的那样:“李鸿章第一认定我国到了十九世纪惟有学西洋的科学机械然后能生存。第二,李鸿章在同治三年已经看清中国与日本,孰强孰弱,要看哪一国变的快。日本明治维新运动的世界的历史的意义,他一下就看清了,并且大声疾呼的要当时的人猛醒与努力。第三,李鸿章认定改革要从培养人才下手,所以他要改革前清的科举制度。不但此也;他简直要改革士大夫的人生观。他要士大夫放弃章句小楷之积习,而把科学工程悬为终身富贵的鹄的。”⑨ 总的来说,李鸿章之流就是要“讲求洋器”,平定发捻,自立自强,抵御外患,重点是放在购船、造炮、练兵等军事方面。“以中国之伦常名教为原本,辅以诸国富强之术。”这一指导思想就成为此些人等的大兴办理洋务的纲领,并历时三十余年而不衰。而正是因为李鸿章认识时代之清楚,所以他成了同治、光绪年间自强运动的中心人物。
	\end{frame}
	\begin{frame}
		从19世纪60年代初始,洋务运动以“求强”、“求富”为目的,提倡在军事、工矿企业、交通运输和文化教育等诸领域中,向西方国家学习,在中国走向近代的道路上跨出了较大的一步。但是在半殖民半封建社会的历史条件下,西方帝国主义列强决不会愿意也不可能允许中国通过兴办洋务富强起来。因此他们在表面上扶植、支持洋务的同时,又不断采取政治的、经济的、外交的乃至军事的手段进行侵略和控制。而作为洋务派的代表人物李鸿章始终对外国侵略者心存畏惧,总想以妥协退让换取与外国侵略者相安无事,结果只能是处处碰壁。
	\end{frame}
%%%%%%%%%%%%%%%%%%%%%%%%%%%%%%%%%%%%%%%%%%%%%%%%%%%%%%%%%%%%%%%%%%%%%%%%%%%%%%%%%%%%%%%%%%%%%%%	
	
	%参考文献(非调用.bib文件,而是手动输入)
	\appendix
	\begin{frame}{参考文献}
		\begin{thebibliography}{99} % 最大可能的参考文献数目,可以根据实际情况调整
			\bibitem[Author et al., 2021]{ref1}
			Dewey. Art as Experience[J]. 高等教育出版社, 11.4(2022):1-4.
			\newblock Title of the first reference.
			\newblock \emph{Journal Name}, \emph{Volume}(Issue), PageRange.
			
			\bibitem[Another Author, 2022]{ref2}
			 Plato. Utopia[M]. 高等教育出版社, 01(2333):-2-4.
			\newblock Title of the second reference.
			\newblock \emph{Another Journal}, \emph{Volume}(Issue), PageRange.
		\end{thebibliography}
	\end{frame}
	
	\begin{frame}[plain,c]
		\begin{center}
			\Huge 感谢聆听 !
		\end{center}
	\end{frame}
	
\end{document}