\documentclass[10pt,a4paper]{beamer} %ppt模式

\usepackage{ctex}

\usepackage{beamerthemesplit} % 加载主题宏包
\usetheme{Warsaw} % 选用该主题

\usepackage[T1]{fontenc}

%插入图片, 写公式, 画表格等
\usepackage{subfig}
\usepackage{amssymb,amsmath,mathtools}
\usepackage{amsfonts,booktabs}
\usepackage{lmodern,textcomp}
\usepackage{color}
\usepackage{tikz}
\usepackage[utf8]{inputenc}
\usepackage{natbib}
\usepackage{multicol}

\setbeamertemplate{footline}{}%清除底部信息,更美观

\begin{document}
	
	%基本信息
	\title{对杜威“感觉是直接呈现在经验\\之中不证自明的意义”观点的评说}
	\subtitle{《艺术即经验》导读课程第三次报告}
	\author{林海轩}
	\institute{复旦大学物理学系}
	\date{}
	
	%生成标题页
	\begin{frame}
		\titlepage
	\end{frame}
	
	%目录页,代码无需改动
	\begin{frame}
		\tableofcontents
	\end{frame}
	
%%%%%%%%%%%%%%%%%%%%%%%%%%%%%%%%%%%%%%%%%%%%%%%%%%%%%%%%%%%%%%%%%%%%%%%%%%%%%%%%%%%%%%%%%%%%%%%
	\section{引言}
	\begin{frame}
		\quad\quad 杜威强调感觉是直接呈现在经验中不证自明的
		意义,这一观点突显了感觉在我们对世界的认知和
		艺术体验中的基础作用。然而,这一观点也引发了
		深刻的思考和潜在的争议。本文将对杜威的观点展
		开深入讨论,探讨感觉的直观性、复杂性以及与认
		知、文化的关系,同时结合生活例子和文学经典,
		对这一观点进行拓展。
	\end{frame}
	
	\section{正文}
		\subsection{论述感觉的直观性}
		\begin{frame}
			\quad\quad 首先,值得肯定的是,感觉确实在我们日常经
			验中起着至关重要的作用。无论是看到一幅画、听
			到一首音乐,还是触摸到某物,感觉是我们获取信
			息、理解世界的最直接途径之一。杜威的观点突显
			了感觉的即时性和不需要经过推理的特性,这与艺
			术创作和欣赏过程中强调的直觉体验紧密相连。
			我们可以从日常生活中找到许多例子来支持感
			觉的直观性。比如,当我们看到一片蓝天或感受到
			阳光的温暖时,这些感觉无需经过深思熟虑,直接
			而即时地构成了我们的经验。在艺术中,画家通过
			色彩的运用,音乐家通过音符的组合,都在追求观
			众在感觉上的直接共鸣。这种直观性使得感觉成为
			艺术创作和欣赏的核心元素。
		\end{frame}
		\subsection{反驳——感觉的意义可能并非完全不证自明}
		\begin{frame}
			\quad\quad 然而,我们也需要注意到,感觉并非是一个孤
			立的元素,而是与认知、文化、个体经历等因素交
			织在一起的。在一定程度上,感觉的意义可能并非
			完全不证自明,而可能受到认知框架、文化传统和
			个体经验的影响。例如,在不同的文化背景中,对
			于某种颜色或图像的感觉可能会产生不同的解释和
			情感反应。这一点使得感觉的直观性和不证自明性
			变得相对而言,而非绝对。
		
			\quad\quad 从认知科学的角度来看,感觉与认知过程之间
			存在着复杂的相互作用。感觉是认知的一部分,但
			认知也可以调节感觉的解释和理解。这种相互作用
			表明,感觉不仅仅是被动接受的信息,而可能受到
			个体思维和认知架构的主动塑造。
		
			\quad\quad 借用康德的观点,他在《纯粹理性批判》中强
			调经验的直观性和先验性。康德认为经验始于感
			觉,而这种感觉又是直接而无需推理的。然而,康
			德也指出,认知的先验结构对感觉的解释起着至关
			重要的作用。在文学作品中,我们常常看到作家通
			过描绘感官细节,以唤起读者的感觉体验。但这些
			感觉又往往受到文学结构和主题的塑造,由此可以
			看出感觉与认知之间的错综复杂的关系。
		\end{frame}
		\subsection{从其他领域的视角阐释感觉}
		\begin{frame}
			\quad\quad 在艺术领域,尤其是在抽象艺术或实验性艺术
			的情境下,感觉的解释可能更加开放和主观。观众
			在欣赏艺术品时可能经历着不同的感觉体验,而这
			些体验往往是复杂多样的,涉及到情感、记忆和审
			美情趣。这种多样性可能表明感觉在艺术中的意义
			并非单一而固定的,而是具有灵活性和个体化的特
			点。
		
			\quad\quad 从进化的角度看,感觉系统的发展和演化是为
			了帮助生物适应环境、获取生存所需的信息。这一
			生存需要的直觉性可能构成了感觉不证自明性的一
			部分。比如,当我们感知到某物的危险性,身体会
			自发地产生紧张和警觉,这种感觉有助于我们的自
			我保护。这一进化的视角提供了感觉直观性的一种
			基础解释,即感觉的直观性在进化中可能具有生存
			的适应性。
		\end{frame}
	\section{总结}
	\begin{frame}
		\quad\quad 综合而言,杜威关于感觉在经验中不证自明的
		意义的观点提供了一个重要的思考角度。感觉的直
		观性在艺术和日常生活中都具有显著作用,但我们
		也需要认识到感觉与认知、文化之间的复杂相互作
		用。生活例子和文学经典的引证为我们提供了更具
		体的论据,进一步拓展了对这一观点的理解。感觉
		的多样性和进化的角度为我们提供了更全面、多层
		次的视角,使我们更好地理解感觉在塑造我们对世
		界的认知和艺术体验中的作用。这一探讨不仅有助
		于深化对感觉的理解,也为艺术哲学和认知科学的
		交叉领域提供了丰富的思考空间。
	\end{frame}
%%%%%%%%%%%%%%%%%%%%%%%%%%%%%%%%%%%%%%%%%%%%%%%%%%%%%%%%%%%%%%%%%%%%%%%%%%%%%%%%%%%%%%%%%%%%%%%	
	
	%参考文献(非调用.bib文件,而是手动输入)
	\appendix
	\begin{frame}{参考文献}
		\begin{thebibliography}{99} % 最大可能的参考文献数目,可以根据实际情况调整
			\bibitem[Author et al., 2021]{ref1}
			Dewey. Art as Experience[J]. 高等教育出版社, 11.4(2022):1-4.
			\newblock Title of the first reference.
			\newblock \emph{Journal Name}, \emph{Volume}(Issue), PageRange.
			
			\bibitem[Another Author, 2022]{ref2}
			 Plato. Utopia[M]. 高等教育出版社, 01(2333):-2-4.
			\newblock Title of the second reference.
			\newblock \emph{Another Journal}, \emph{Volume}(Issue), PageRange.
		\end{thebibliography}
	\end{frame}
	
	\begin{frame}[plain,c]
		\begin{center}
			\Huge 感谢聆听 !
		\end{center}
	\end{frame}
	
\end{document}