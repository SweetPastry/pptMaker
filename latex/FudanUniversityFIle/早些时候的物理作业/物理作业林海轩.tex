
%%%%%%%%%%%%%%%%%%%%%%%%%%%%%%%%%%%%%%%%%%%%%%%%%%%%%%%%
\documentclass[12pt,a4paper,twocolumn]{article}% 文档格式
\usepackage{ctex,hyperref}% 输出汉字
\usepackage{times}% 英文使用Times New Roman
%%%%%%%%%%%%%%%%%%%%%%%%%%%%%%%%%%%%%%%%%%%%%%%%%%%%%%%%
\title{\fontsize{18pt}{27pt}\selectfont% 小四字号,1.5倍行距
	{\heiti% 黑体 
		Physics Homework}}% 题目
%%%%%%%%%%%%%%%%%%%%%%%%%%%%%%%%%%%%%%%%%%%%%%%%%%%%%%%%
\author{\fontsize{12pt}{18pt}\selectfont% 小四字号,1.5倍行距
	{\fangsong% 仿宋
	Haixuan Lin}\\% 标题栏脚注
	\fontsize{10.5pt}{15.75pt}\selectfont% 五号字号,1.5倍行距
	{\fangsong% 仿宋
		(Fudan University department of physics
		)}}% 作者单位,“~”表示空格
%%%%%%%%%%%%%%%%%%%%%%%%%%%%%%%%%%%%%%%%%%%%%%%%%%%%%%%%
\date{}% 日期(这里避免生成日期)
%%%%%%%%%%%%%%%%%%%%%%%%%%%%%%%%%%%%%%%%%%%%%%%%%%%%%%%%
\usepackage{amsmath,amsfonts,amssymb}% 为公式输入创造条件的宏包
%%%%%%%%%%%%%%%%%%%%%%%%%%%%%%%%%%%%%%%%%%%%%%%%%%%%%%%%
\usepackage{graphicx}% 图片插入宏包
\usepackage{subfigure}% 并排子图
\usepackage{float}% 浮动环境,用于调整图片位置
\usepackage[export]{adjustbox}% 防止过宽的图片
%%%%%%%%%%%%%%%%%%%%%%%%%%%%%%%%%%%%%%%%%%%%%%%%%%%%%%%%
\usepackage{bibentry}
\usepackage{natbib}% 以上2个为参考文献宏包
%%%%%%%%%%%%%%%%%%%%%%%%%%%%%%%%%%%%%%%%%%%%%%%%%%%%%%%%
\usepackage{abstract}% 两栏文档,一栏摘要及关键字宏包
\renewcommand{\abstracttextfont}{\fangsong}% 摘要内容字体为仿宋
\renewcommand{\abstractname}{\textbf{摘\quad 要}}% 更改摘要二字的样式
%%%%%%%%%%%%%%%%%%%%%%%%%%%%%%%%%%%%%%%%%%%%%%%%%%%%%%%%
\usepackage{xcolor}% 字体颜色宏包
\newcommand{\red}[1]{\textcolor[rgb]{1.00,0.00,0.00}{#1}}
\newcommand{\blue}[1]{\textcolor[rgb]{0.00,0.00,1.00}{#1}}
\newcommand{\green}[1]{\textcolor[rgb]{0.00,1.00,0.00}{#1}}
\newcommand{\darkblue}[1]
{\textcolor[rgb]{0.00,0.00,0.50}{#1}}
\newcommand{\darkgreen}[1]
{\textcolor[rgb]{0.00,0.37,0.00}{#1}}
\newcommand{\darkred}[1]{\textcolor[rgb]{0.60,0.00,0.00}{#1}}
\newcommand{\brown}[1]{\textcolor[rgb]{0.50,0.30,0.00}{#1}}
\newcommand{\purple}[1]{\textcolor[rgb]{0.50,0.00,0.50}{#1}}% 为使用方便而编辑的新指令
%%%%%%%%%%%%%%%%%%%%%%%%%%%%%%%%%%%%%%%%%%%%%%%%%%%%%%%%
\usepackage{url}% 超链接
\usepackage{bm}% 加粗部分公式
\usepackage{multirow}
\usepackage{booktabs}
\usepackage{epstopdf}
\usepackage{epsfig}
\usepackage{longtable}% 长表格
\usepackage{supertabular}% 跨页表格
\usepackage{algorithm}
\usepackage{algorithmic}
\usepackage{changepage}% 换页
%%%%%%%%%%%%%%%%%%%%%%%%%%%%%%%%%%%%%%%%%%%%%%%%%%%%%%%%
\usepackage{enumerate}% 短编号
\usepackage{caption}% 设置标题
\captionsetup[figure]{name=\fontsize{10pt}{15pt}\selectfont Figure}% 设置图片编号头
\captionsetup[table]{name=\fontsize{10pt}{15pt}\selectfont Table}% 设置表格编号头
%%%%%%%%%%%%%%%%%%%%%%%%%%%%%%%%%%%%%%%%%%%%%%%%%%%%%%%%
\usepackage{indentfirst}% 中文首行缩进
\usepackage[left=2.50cm,right=2.50cm,top=2.80cm,bottom=2.50cm]{geometry}% 页边距设置
\renewcommand{\baselinestretch}{1.5}% 定义行间距(1.5)
%%%%%%%%%%%%%%%%%%%%%%%%%%%%%%%%%%%%%%%%%%%%%%%%%%%%%%%%
\usepackage{fancyhdr} %设置全文页眉、页脚的格式
\pagestyle{fancy}
\hypersetup{colorlinks=true,linkcolor=black}% 去除引用红框,改变颜色
%%%%%%%%%%%%%%%%%%%%%%%%%%%%%%%%%%%%%%%%%%%%%%%%%%%%%%%%
\usepackage{caption}%标题取消自动figure
\usepackage{multicol}
\usepackage{cuted}
%%%%%%%%%%%%%%%%%%%%%%%%%%%%%%%%%%%%%%%%%%%%%%%%%%%%%%%%
\begin{document}% 以下为正文内容
	\maketitle% 产生标题,没有它无法显示标题
	%%%%%%%%%%%%%%%%%%%%%%%%%%%%%%%%%%%%%%%%%%%%%%%%%%%%%%%%
	\lhead{}% 页眉左边设为空
	\chead{}% 页眉中间设为空
	\rhead{}% 页眉右边设为空
	\lfoot{}% 页脚左边设为空
	\cfoot{\thepage}% 页脚中间显示页码
	\rfoot{}% 页脚右边设为空
	%%%%%%%%%%%%%%%%%%%%%%%%%%%%%%%%%%%%%%%%%%%%%%%%%%%%%%%%
	
	\begin{strip}	
	\begin{center}% 居中处理
		{\textbf{Abstract}}% 英文摘要
	\end{center}
	\begin{adjustwidth}{1.06cm}{1.06cm}% 英文摘要内容
		\hspace{1.5em}In order to improve my computer and English skills, please allow me to complete this physics homework in English context with \LaTeX, so as to improve my professional level. Sorry for the inconvenience!
	\end{adjustwidth}
	\end{strip}
	
	
	\clearpage% 从新的一页继续
	
	
	
		\section{Question 5-2}
		\noindent For satellite, according to the law of conservation of angular momentum, we have
		\begin{equation}
			3R\cdot m4v_0 = r\cdot mv_0
		\end{equation}
		So $$r=12R$$
		
		\section{Question 5-5}
		\subsection*{(1)}
		\noindent Before and after a small rocket launch, the system's orbital energy is
		\begin{equation}
			\frac{1}{2}{mv_0^2}-G\frac{Mm}{R_0}=-G\frac{Mm}{2R_0}
		\end{equation}
		\begin{equation}
			\frac{1}{2}m\left( v_0+\Delta v_1 \right) ^2-G\frac{Mm}{R_0}=-G\frac{Mm}{\frac{8}{3}R_0}
		\end{equation}
		So$$\frac{\Delta v_1}{v_0}=\frac{\sqrt{5}}{2}-1$$
		\subsection*{(2)}
		\noindent In order to accomplish the same effect, $\Delta v_2$ subjuets to
		\begin{equation}
			\frac{1}{2}m\left( v_0+\Delta v_1 \right) ^2=\frac{1}{2}m\left( {v_0^2}+\Delta {v_2^2} \right) 
		\end{equation}
		So $$\frac{\Delta v_2}{v_0}=\frac{1}{2}$$
		\section{Question 5-8}
		\noindent Let $J$ be the moment of inertia of the rod, according to the law of conservation of angular momentum, we have
		\begin{equation}
			J\omega=mv_0l
		\end{equation}
		The mechanical energy of the system is conserved, so
		\begin{equation}
			\frac{1}{2}J\omega ^2=\left( mgl+m^\prime2l \right) \left( 1-\cos \theta \right)
		\end{equation}
		Finally
		\begin{equation}
			J=m^\prime l^2+\left( m+m^\prime \right) l^2+m^\prime4l^2
		\end{equation}
		So $$\theta =\mathrm{arc}\cos \left( 1-\frac{{v_0^2}}{2g}\frac{m^2}{\left( m+6m^{\prime} \right) \left( m+2m^{\prime} \right)} \right) $$
		\begin{figure*}[t]
			\centering
			\includegraphics[width=0.7\linewidth]{../Latex/5-10}
			\caption*{Figure for 5-10}
			\label{fig:5-10}
		\end{figure*}
		\section{Question 5-9}
		\subsection*{(1)}
		\noindent According to the defination of barycenter, we choose the middle point of the rope and
		\begin{equation}
			x_C=\frac{m_1\frac{-l}{2}+m_2\frac{l}{2}}{m_1+m_2}
		\end{equation}
		The above equation takes a derivative with respect to time
		\begin{equation}
			v_C=\frac{m_1v_1+m_2v_2}{m_1+m_2}
		\end{equation}
		In the system of barycenter
		\begin{equation}
			v_1^\prime=v_1-v_C
		\end{equation}
		\begin{equation}
			\omega=\frac{v_1^\prime}{x_C+\frac{l}{2}}
		\end{equation}
		So
		\begin{align*}
		L&=[m_1(x_C+\frac{l}{2})^2+m_2(\frac{l}{2}-x_C)^2]\omega    \\ &=\frac{m_1m_2}{m_1+m_2}(v_1+v_2)l
		\end{align*}
		\subsection*{(2)}
		\noindent In the system of barycenter, tension provides centripetal force
		\begin{align*}
			T=m_1\omega^2r_1=\frac{m_1m_2}{m_1+m_2}\frac{(v_1-v_2)^2}{l}
		\end{align*}
		
		
		
		\section{Question 5-10}
		

		
		\subsection*{(1)}
		\noindent Let $\beta$ subjuects to
		\begin{equation}
			v_f=v_0tan\beta
		\end{equation}
		And impulse $I$ subjects to
		\begin{equation}
			Icos\beta=mv_0
		\end{equation}
		For the system on the bar, according to the impulse theorem
		\begin{equation}
			2mv_C=I
		\end{equation}
		According to theorem of moment of impulse, for the system on the bar
		\begin{equation}
			J\omega=I\frac{l}{2}sin(\frac{\pi}{4}-\beta)
		\end{equation}
		Because it's an elastic collision, mechanical energy is conserved
		\begin{equation}
			\frac{1}{2}{mv_0^2}=\frac{1}{2}{mv_f^2}+\frac{1}{2}2{mv_C^2}+\frac{1}{2}J\omega ^2
		\end{equation}
		So 
		\begin{align*}
			v_f&=v_0tan\left(\frac{1}{2}arctan\frac{51\sqrt{2}+34}{68}\right) \\
			&\approx0.5469181607v_0
		\end{align*}
		\subsection*{(2)}
		\noindent We can also have that
		\begin{align*}
			\omega&=\frac{v_0}{\sqrt{2}l}\left[1-tan\left(\frac{1}{2}arctan\frac{51\sqrt{2}+34}{68}\right)\right]\\
			&\approx 0.320377241\frac{v_0}{l}
		\end{align*}
		\section{Question 5-11}
		\noindent According to Newton's second law and kinematics
		\begin{equation}
			T=m_1g+m_1a_1=m_2g+m_2a_2
		\end{equation}
		\begin{equation}
			t=\sqrt{\frac{2h_1}{g}}=\sqrt{\frac{2h_2}{g}}
		\end{equation}
		The solution is that
		$$
		t=\sqrt{\frac{-2}{g}\frac{m_1h_1-m_2h_2}{m_1-m_2}}
		$$
		\section{Question 5-15}
		\noindent By observation, $mg<m\frac{v_0^2}{r_0}=\frac{9}{2}mg$, as a result, the object will go futher and futher from the centre
		$$
		r_{min}=r_0
		$$
		In this process, according to conservation of angular momentum and conservation of mechanical energy
		\begin{equation}
			mvr=mv_0r_0
		\end{equation}
		\begin{equation}
			\frac{1}{2}m\left( v_{0}^{2}-v^2 \right) =mg\left( r-r_0 \right) 
		\end{equation}
		So $$r_{max}=3r_0$$
		\section{Question 5-19}
		\noindent According to conservation of angular momentum and conservation of mechanical energy
		\begin{equation}
			2ma^2\omega _0=2mx^2\omega 
		\end{equation}
		\begin{equation}
			\frac{1}{2}2ma^2\omega _{0}^{2}=\frac{1}{2}2mx^2\omega ^2+2\frac{1}{2}m\left( \frac{d}{dt}x \right) ^2
		\end{equation}
		The $x$ is the distance from the centre to the small ball. The solution can be] write as below
		$$
		\omega=\frac{\omega_0}{\omega_0^2t^2+1}
		$$
		$$
		\alpha=\frac{d}{dt}\omega=-\frac{2\omega_0^2t}{(\omega_0^2t^2+1)^2}
		$$
		\section{Question 5-20}
		\subsection*{(1)}
		\noindent According to conservation of angular momentum and conservation of mechanical energy
		\begin{equation}
			mva=mv_1L_1
		\end{equation}
		\begin{equation}
			3mva=mv_2(4a-L_1)
		\end{equation}
		\begin{equation}
			2\frac{1}{2}mv^2=\frac{1}{2}mv_1^2+\frac{1}{2}mv_2^2
		\end{equation}
		So 
		\begin{align*}
				L_1&=\frac{6}{\sqrt{46}\cos \left( \frac{1}{3}\mathrm{arc}\cos \left( -\frac{121}{23\sqrt{46}} \right) \right) -1}a \\
				&\approx1.653165286a
		\end{align*}
		\subsection*{(2)}
		\noindent According to conservation of angular momentum and conservation of mechanical energy
		\begin{equation}
			mrv_{1n}=mav
		\end{equation}
		\begin{equation}
			m(4a-r)v_{2n}=m3av
		\end{equation}
		The length of the rope is constant
		\begin{equation}
			\frac{d^2r_1}{dt^2}+\frac{d^2r_2}{dt^2}=0
		\end{equation}
		The kinetic equations are
		\begin{equation}
			m\frac{d^2r_1}{dt^2}=m\frac{v_{1n}^2}{r}-T
		\end{equation}
		\begin{equation}
			m\frac{d^2r_2}{dt^2}=m\frac{v_{2n}^2}{r}-T
		\end{equation}
		We can get the expression of $T$, and according to $Holder$ inequation
		\begin{align*}
		T&=\frac{ma^2v^2}{2}\left( \frac{1}{r^3}+\frac{3}{\left( 4a-r \right) ^3} \right) \\
		&\geqslant \frac{ma^2v^2}{2}\frac{\left( 1+\sqrt{3} \right) ^4}{\left( r+4a-r \right) ^3} \\
		&=\frac{\left( 1+\sqrt{3} \right) ^4}{128}\frac{mv^2}{a} \\
		&\approx 0.4352563509\frac{mv^2}{a}
		\end{align*}
		
		
\end{document}% 结束文档编辑,后面写啥都编译不出来